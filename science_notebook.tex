
    




    
\documentclass[11pt]{article}

    
    \usepackage[breakable]{tcolorbox}
    \tcbset{nobeforeafter} % prevents tcolorboxes being placing in paragraphs
    \usepackage{float}
    \floatplacement{figure}{H} % forces figures to be placed at the correct location
    
    \usepackage[T1]{fontenc}
    % Nicer default font (+ math font) than Computer Modern for most use cases
    \usepackage{mathpazo}

    % Basic figure setup, for now with no caption control since it's done
    % automatically by Pandoc (which extracts ![](path) syntax from Markdown).
    \usepackage{graphicx}
    % We will generate all images so they have a width \maxwidth. This means
    % that they will get their normal width if they fit onto the page, but
    % are scaled down if they would overflow the margins.
    \makeatletter
    \def\maxwidth{\ifdim\Gin@nat@width>\linewidth\linewidth
    \else\Gin@nat@width\fi}
    \makeatother
    \let\Oldincludegraphics\includegraphics
    % Set max figure width to be 80% of text width, for now hardcoded.
    \renewcommand{\includegraphics}[1]{\Oldincludegraphics[width=.8\maxwidth]{#1}}
    % Ensure that by default, figures have no caption (until we provide a
    % proper Figure object with a Caption API and a way to capture that
    % in the conversion process - todo).
    \usepackage{caption}
    \DeclareCaptionLabelFormat{nolabel}{}
    \captionsetup{labelformat=nolabel}

    \usepackage{adjustbox} % Used to constrain images to a maximum size 
    \usepackage{xcolor} % Allow colors to be defined
    \usepackage{enumerate} % Needed for markdown enumerations to work
    \usepackage{geometry} % Used to adjust the document margins
    \usepackage{amsmath} % Equations
    \usepackage{amssymb} % Equations
    \usepackage{textcomp} % defines textquotesingle
    % Hack from http://tex.stackexchange.com/a/47451/13684:
    \AtBeginDocument{%
        \def\PYZsq{\textquotesingle}% Upright quotes in Pygmentized code
    }
    \usepackage{upquote} % Upright quotes for verbatim code
    \usepackage{eurosym} % defines \euro
    \usepackage[mathletters]{ucs} % Extended unicode (utf-8) support
    \usepackage[utf8x]{inputenc} % Allow utf-8 characters in the tex document
    \usepackage{fancyvrb} % verbatim replacement that allows latex
    \usepackage{grffile} % extends the file name processing of package graphics 
                         % to support a larger range 
    % The hyperref package gives us a pdf with properly built
    % internal navigation ('pdf bookmarks' for the table of contents,
    % internal cross-reference links, web links for URLs, etc.)
    \usepackage{hyperref}
    \usepackage{longtable} % longtable support required by pandoc >1.10
    \usepackage{booktabs}  % table support for pandoc > 1.12.2
    \usepackage[inline]{enumitem} % IRkernel/repr support (it uses the enumerate* environment)
    \usepackage[normalem]{ulem} % ulem is needed to support strikethroughs (\sout)
                                % normalem makes italics be italics, not underlines
    \usepackage{mathrsfs}
    

    
    % Colors for the hyperref package
    \definecolor{urlcolor}{rgb}{0,.145,.698}
    \definecolor{linkcolor}{rgb}{.71,0.21,0.01}
    \definecolor{citecolor}{rgb}{.12,.54,.11}

    % ANSI colors
    \definecolor{ansi-black}{HTML}{3E424D}
    \definecolor{ansi-black-intense}{HTML}{282C36}
    \definecolor{ansi-red}{HTML}{E75C58}
    \definecolor{ansi-red-intense}{HTML}{B22B31}
    \definecolor{ansi-green}{HTML}{00A250}
    \definecolor{ansi-green-intense}{HTML}{007427}
    \definecolor{ansi-yellow}{HTML}{DDB62B}
    \definecolor{ansi-yellow-intense}{HTML}{B27D12}
    \definecolor{ansi-blue}{HTML}{208FFB}
    \definecolor{ansi-blue-intense}{HTML}{0065CA}
    \definecolor{ansi-magenta}{HTML}{D160C4}
    \definecolor{ansi-magenta-intense}{HTML}{A03196}
    \definecolor{ansi-cyan}{HTML}{60C6C8}
    \definecolor{ansi-cyan-intense}{HTML}{258F8F}
    \definecolor{ansi-white}{HTML}{C5C1B4}
    \definecolor{ansi-white-intense}{HTML}{A1A6B2}
    \definecolor{ansi-default-inverse-fg}{HTML}{FFFFFF}
    \definecolor{ansi-default-inverse-bg}{HTML}{000000}

    % commands and environments needed by pandoc snippets
    % extracted from the output of `pandoc -s`
    \providecommand{\tightlist}{%
      \setlength{\itemsep}{0pt}\setlength{\parskip}{0pt}}
    \DefineVerbatimEnvironment{Highlighting}{Verbatim}{commandchars=\\\{\}}
    % Add ',fontsize=\small' for more characters per line
    \newenvironment{Shaded}{}{}
    \newcommand{\KeywordTok}[1]{\textcolor[rgb]{0.00,0.44,0.13}{\textbf{{#1}}}}
    \newcommand{\DataTypeTok}[1]{\textcolor[rgb]{0.56,0.13,0.00}{{#1}}}
    \newcommand{\DecValTok}[1]{\textcolor[rgb]{0.25,0.63,0.44}{{#1}}}
    \newcommand{\BaseNTok}[1]{\textcolor[rgb]{0.25,0.63,0.44}{{#1}}}
    \newcommand{\FloatTok}[1]{\textcolor[rgb]{0.25,0.63,0.44}{{#1}}}
    \newcommand{\CharTok}[1]{\textcolor[rgb]{0.25,0.44,0.63}{{#1}}}
    \newcommand{\StringTok}[1]{\textcolor[rgb]{0.25,0.44,0.63}{{#1}}}
    \newcommand{\CommentTok}[1]{\textcolor[rgb]{0.38,0.63,0.69}{\textit{{#1}}}}
    \newcommand{\OtherTok}[1]{\textcolor[rgb]{0.00,0.44,0.13}{{#1}}}
    \newcommand{\AlertTok}[1]{\textcolor[rgb]{1.00,0.00,0.00}{\textbf{{#1}}}}
    \newcommand{\FunctionTok}[1]{\textcolor[rgb]{0.02,0.16,0.49}{{#1}}}
    \newcommand{\RegionMarkerTok}[1]{{#1}}
    \newcommand{\ErrorTok}[1]{\textcolor[rgb]{1.00,0.00,0.00}{\textbf{{#1}}}}
    \newcommand{\NormalTok}[1]{{#1}}
    
    % Additional commands for more recent versions of Pandoc
    \newcommand{\ConstantTok}[1]{\textcolor[rgb]{0.53,0.00,0.00}{{#1}}}
    \newcommand{\SpecialCharTok}[1]{\textcolor[rgb]{0.25,0.44,0.63}{{#1}}}
    \newcommand{\VerbatimStringTok}[1]{\textcolor[rgb]{0.25,0.44,0.63}{{#1}}}
    \newcommand{\SpecialStringTok}[1]{\textcolor[rgb]{0.73,0.40,0.53}{{#1}}}
    \newcommand{\ImportTok}[1]{{#1}}
    \newcommand{\DocumentationTok}[1]{\textcolor[rgb]{0.73,0.13,0.13}{\textit{{#1}}}}
    \newcommand{\AnnotationTok}[1]{\textcolor[rgb]{0.38,0.63,0.69}{\textbf{\textit{{#1}}}}}
    \newcommand{\CommentVarTok}[1]{\textcolor[rgb]{0.38,0.63,0.69}{\textbf{\textit{{#1}}}}}
    \newcommand{\VariableTok}[1]{\textcolor[rgb]{0.10,0.09,0.49}{{#1}}}
    \newcommand{\ControlFlowTok}[1]{\textcolor[rgb]{0.00,0.44,0.13}{\textbf{{#1}}}}
    \newcommand{\OperatorTok}[1]{\textcolor[rgb]{0.40,0.40,0.40}{{#1}}}
    \newcommand{\BuiltInTok}[1]{{#1}}
    \newcommand{\ExtensionTok}[1]{{#1}}
    \newcommand{\PreprocessorTok}[1]{\textcolor[rgb]{0.74,0.48,0.00}{{#1}}}
    \newcommand{\AttributeTok}[1]{\textcolor[rgb]{0.49,0.56,0.16}{{#1}}}
    \newcommand{\InformationTok}[1]{\textcolor[rgb]{0.38,0.63,0.69}{\textbf{\textit{{#1}}}}}
    \newcommand{\WarningTok}[1]{\textcolor[rgb]{0.38,0.63,0.69}{\textbf{\textit{{#1}}}}}
    
    
    % Define a nice break command that doesn't care if a line doesn't already
    % exist.
    \def\br{\hspace*{\fill} \\* }
    % Math Jax compatibility definitions
    \def\gt{>}
    \def\lt{<}
    \let\Oldtex\TeX
    \let\Oldlatex\LaTeX
    \renewcommand{\TeX}{\textrm{\Oldtex}}
    \renewcommand{\LaTeX}{\textrm{\Oldlatex}}
    % Document parameters
    % Document title
    \title{science\_notebook}
    \author{email@allen.tools}
    
    
    
    
% Pygments definitions
\makeatletter
\def\PY@reset{\let\PY@it=\relax \let\PY@bf=\relax%
    \let\PY@ul=\relax \let\PY@tc=\relax%
    \let\PY@bc=\relax \let\PY@ff=\relax}
\def\PY@tok#1{\csname PY@tok@#1\endcsname}
\def\PY@toks#1+{\ifx\relax#1\empty\else%
    \PY@tok{#1}\expandafter\PY@toks\fi}
\def\PY@do#1{\PY@bc{\PY@tc{\PY@ul{%
    \PY@it{\PY@bf{\PY@ff{#1}}}}}}}
\def\PY#1#2{\PY@reset\PY@toks#1+\relax+\PY@do{#2}}

\expandafter\def\csname PY@tok@w\endcsname{\def\PY@tc##1{\textcolor[rgb]{0.73,0.73,0.73}{##1}}}
\expandafter\def\csname PY@tok@c\endcsname{\let\PY@it=\textit\def\PY@tc##1{\textcolor[rgb]{0.25,0.50,0.50}{##1}}}
\expandafter\def\csname PY@tok@cp\endcsname{\def\PY@tc##1{\textcolor[rgb]{0.74,0.48,0.00}{##1}}}
\expandafter\def\csname PY@tok@k\endcsname{\let\PY@bf=\textbf\def\PY@tc##1{\textcolor[rgb]{0.00,0.50,0.00}{##1}}}
\expandafter\def\csname PY@tok@kp\endcsname{\def\PY@tc##1{\textcolor[rgb]{0.00,0.50,0.00}{##1}}}
\expandafter\def\csname PY@tok@kt\endcsname{\def\PY@tc##1{\textcolor[rgb]{0.69,0.00,0.25}{##1}}}
\expandafter\def\csname PY@tok@o\endcsname{\def\PY@tc##1{\textcolor[rgb]{0.40,0.40,0.40}{##1}}}
\expandafter\def\csname PY@tok@ow\endcsname{\let\PY@bf=\textbf\def\PY@tc##1{\textcolor[rgb]{0.67,0.13,1.00}{##1}}}
\expandafter\def\csname PY@tok@nb\endcsname{\def\PY@tc##1{\textcolor[rgb]{0.00,0.50,0.00}{##1}}}
\expandafter\def\csname PY@tok@nf\endcsname{\def\PY@tc##1{\textcolor[rgb]{0.00,0.00,1.00}{##1}}}
\expandafter\def\csname PY@tok@nc\endcsname{\let\PY@bf=\textbf\def\PY@tc##1{\textcolor[rgb]{0.00,0.00,1.00}{##1}}}
\expandafter\def\csname PY@tok@nn\endcsname{\let\PY@bf=\textbf\def\PY@tc##1{\textcolor[rgb]{0.00,0.00,1.00}{##1}}}
\expandafter\def\csname PY@tok@ne\endcsname{\let\PY@bf=\textbf\def\PY@tc##1{\textcolor[rgb]{0.82,0.25,0.23}{##1}}}
\expandafter\def\csname PY@tok@nv\endcsname{\def\PY@tc##1{\textcolor[rgb]{0.10,0.09,0.49}{##1}}}
\expandafter\def\csname PY@tok@no\endcsname{\def\PY@tc##1{\textcolor[rgb]{0.53,0.00,0.00}{##1}}}
\expandafter\def\csname PY@tok@nl\endcsname{\def\PY@tc##1{\textcolor[rgb]{0.63,0.63,0.00}{##1}}}
\expandafter\def\csname PY@tok@ni\endcsname{\let\PY@bf=\textbf\def\PY@tc##1{\textcolor[rgb]{0.60,0.60,0.60}{##1}}}
\expandafter\def\csname PY@tok@na\endcsname{\def\PY@tc##1{\textcolor[rgb]{0.49,0.56,0.16}{##1}}}
\expandafter\def\csname PY@tok@nt\endcsname{\let\PY@bf=\textbf\def\PY@tc##1{\textcolor[rgb]{0.00,0.50,0.00}{##1}}}
\expandafter\def\csname PY@tok@nd\endcsname{\def\PY@tc##1{\textcolor[rgb]{0.67,0.13,1.00}{##1}}}
\expandafter\def\csname PY@tok@s\endcsname{\def\PY@tc##1{\textcolor[rgb]{0.73,0.13,0.13}{##1}}}
\expandafter\def\csname PY@tok@sd\endcsname{\let\PY@it=\textit\def\PY@tc##1{\textcolor[rgb]{0.73,0.13,0.13}{##1}}}
\expandafter\def\csname PY@tok@si\endcsname{\let\PY@bf=\textbf\def\PY@tc##1{\textcolor[rgb]{0.73,0.40,0.53}{##1}}}
\expandafter\def\csname PY@tok@se\endcsname{\let\PY@bf=\textbf\def\PY@tc##1{\textcolor[rgb]{0.73,0.40,0.13}{##1}}}
\expandafter\def\csname PY@tok@sr\endcsname{\def\PY@tc##1{\textcolor[rgb]{0.73,0.40,0.53}{##1}}}
\expandafter\def\csname PY@tok@ss\endcsname{\def\PY@tc##1{\textcolor[rgb]{0.10,0.09,0.49}{##1}}}
\expandafter\def\csname PY@tok@sx\endcsname{\def\PY@tc##1{\textcolor[rgb]{0.00,0.50,0.00}{##1}}}
\expandafter\def\csname PY@tok@m\endcsname{\def\PY@tc##1{\textcolor[rgb]{0.40,0.40,0.40}{##1}}}
\expandafter\def\csname PY@tok@gh\endcsname{\let\PY@bf=\textbf\def\PY@tc##1{\textcolor[rgb]{0.00,0.00,0.50}{##1}}}
\expandafter\def\csname PY@tok@gu\endcsname{\let\PY@bf=\textbf\def\PY@tc##1{\textcolor[rgb]{0.50,0.00,0.50}{##1}}}
\expandafter\def\csname PY@tok@gd\endcsname{\def\PY@tc##1{\textcolor[rgb]{0.63,0.00,0.00}{##1}}}
\expandafter\def\csname PY@tok@gi\endcsname{\def\PY@tc##1{\textcolor[rgb]{0.00,0.63,0.00}{##1}}}
\expandafter\def\csname PY@tok@gr\endcsname{\def\PY@tc##1{\textcolor[rgb]{1.00,0.00,0.00}{##1}}}
\expandafter\def\csname PY@tok@ge\endcsname{\let\PY@it=\textit}
\expandafter\def\csname PY@tok@gs\endcsname{\let\PY@bf=\textbf}
\expandafter\def\csname PY@tok@gp\endcsname{\let\PY@bf=\textbf\def\PY@tc##1{\textcolor[rgb]{0.00,0.00,0.50}{##1}}}
\expandafter\def\csname PY@tok@go\endcsname{\def\PY@tc##1{\textcolor[rgb]{0.53,0.53,0.53}{##1}}}
\expandafter\def\csname PY@tok@gt\endcsname{\def\PY@tc##1{\textcolor[rgb]{0.00,0.27,0.87}{##1}}}
\expandafter\def\csname PY@tok@err\endcsname{\def\PY@bc##1{\setlength{\fboxsep}{0pt}\fcolorbox[rgb]{1.00,0.00,0.00}{1,1,1}{\strut ##1}}}
\expandafter\def\csname PY@tok@kc\endcsname{\let\PY@bf=\textbf\def\PY@tc##1{\textcolor[rgb]{0.00,0.50,0.00}{##1}}}
\expandafter\def\csname PY@tok@kd\endcsname{\let\PY@bf=\textbf\def\PY@tc##1{\textcolor[rgb]{0.00,0.50,0.00}{##1}}}
\expandafter\def\csname PY@tok@kn\endcsname{\let\PY@bf=\textbf\def\PY@tc##1{\textcolor[rgb]{0.00,0.50,0.00}{##1}}}
\expandafter\def\csname PY@tok@kr\endcsname{\let\PY@bf=\textbf\def\PY@tc##1{\textcolor[rgb]{0.00,0.50,0.00}{##1}}}
\expandafter\def\csname PY@tok@bp\endcsname{\def\PY@tc##1{\textcolor[rgb]{0.00,0.50,0.00}{##1}}}
\expandafter\def\csname PY@tok@fm\endcsname{\def\PY@tc##1{\textcolor[rgb]{0.00,0.00,1.00}{##1}}}
\expandafter\def\csname PY@tok@vc\endcsname{\def\PY@tc##1{\textcolor[rgb]{0.10,0.09,0.49}{##1}}}
\expandafter\def\csname PY@tok@vg\endcsname{\def\PY@tc##1{\textcolor[rgb]{0.10,0.09,0.49}{##1}}}
\expandafter\def\csname PY@tok@vi\endcsname{\def\PY@tc##1{\textcolor[rgb]{0.10,0.09,0.49}{##1}}}
\expandafter\def\csname PY@tok@vm\endcsname{\def\PY@tc##1{\textcolor[rgb]{0.10,0.09,0.49}{##1}}}
\expandafter\def\csname PY@tok@sa\endcsname{\def\PY@tc##1{\textcolor[rgb]{0.73,0.13,0.13}{##1}}}
\expandafter\def\csname PY@tok@sb\endcsname{\def\PY@tc##1{\textcolor[rgb]{0.73,0.13,0.13}{##1}}}
\expandafter\def\csname PY@tok@sc\endcsname{\def\PY@tc##1{\textcolor[rgb]{0.73,0.13,0.13}{##1}}}
\expandafter\def\csname PY@tok@dl\endcsname{\def\PY@tc##1{\textcolor[rgb]{0.73,0.13,0.13}{##1}}}
\expandafter\def\csname PY@tok@s2\endcsname{\def\PY@tc##1{\textcolor[rgb]{0.73,0.13,0.13}{##1}}}
\expandafter\def\csname PY@tok@sh\endcsname{\def\PY@tc##1{\textcolor[rgb]{0.73,0.13,0.13}{##1}}}
\expandafter\def\csname PY@tok@s1\endcsname{\def\PY@tc##1{\textcolor[rgb]{0.73,0.13,0.13}{##1}}}
\expandafter\def\csname PY@tok@mb\endcsname{\def\PY@tc##1{\textcolor[rgb]{0.40,0.40,0.40}{##1}}}
\expandafter\def\csname PY@tok@mf\endcsname{\def\PY@tc##1{\textcolor[rgb]{0.40,0.40,0.40}{##1}}}
\expandafter\def\csname PY@tok@mh\endcsname{\def\PY@tc##1{\textcolor[rgb]{0.40,0.40,0.40}{##1}}}
\expandafter\def\csname PY@tok@mi\endcsname{\def\PY@tc##1{\textcolor[rgb]{0.40,0.40,0.40}{##1}}}
\expandafter\def\csname PY@tok@il\endcsname{\def\PY@tc##1{\textcolor[rgb]{0.40,0.40,0.40}{##1}}}
\expandafter\def\csname PY@tok@mo\endcsname{\def\PY@tc##1{\textcolor[rgb]{0.40,0.40,0.40}{##1}}}
\expandafter\def\csname PY@tok@ch\endcsname{\let\PY@it=\textit\def\PY@tc##1{\textcolor[rgb]{0.25,0.50,0.50}{##1}}}
\expandafter\def\csname PY@tok@cm\endcsname{\let\PY@it=\textit\def\PY@tc##1{\textcolor[rgb]{0.25,0.50,0.50}{##1}}}
\expandafter\def\csname PY@tok@cpf\endcsname{\let\PY@it=\textit\def\PY@tc##1{\textcolor[rgb]{0.25,0.50,0.50}{##1}}}
\expandafter\def\csname PY@tok@c1\endcsname{\let\PY@it=\textit\def\PY@tc##1{\textcolor[rgb]{0.25,0.50,0.50}{##1}}}
\expandafter\def\csname PY@tok@cs\endcsname{\let\PY@it=\textit\def\PY@tc##1{\textcolor[rgb]{0.25,0.50,0.50}{##1}}}

\def\PYZbs{\char`\\}
\def\PYZus{\char`\_}
\def\PYZob{\char`\{}
\def\PYZcb{\char`\}}
\def\PYZca{\char`\^}
\def\PYZam{\char`\&}
\def\PYZlt{\char`\<}
\def\PYZgt{\char`\>}
\def\PYZsh{\char`\#}
\def\PYZpc{\char`\%}
\def\PYZdl{\char`\$}
\def\PYZhy{\char`\-}
\def\PYZsq{\char`\'}
\def\PYZdq{\char`\"}
\def\PYZti{\char`\~}
% for compatibility with earlier versions
\def\PYZat{@}
\def\PYZlb{[}
\def\PYZrb{]}
\makeatother


    % For linebreaks inside Verbatim environment from package fancyvrb. 
    \makeatletter
        \newbox\Wrappedcontinuationbox 
        \newbox\Wrappedvisiblespacebox 
        \newcommand*\Wrappedvisiblespace {\textcolor{red}{\textvisiblespace}} 
        \newcommand*\Wrappedcontinuationsymbol {\textcolor{red}{\llap{\tiny$\m@th\hookrightarrow$}}} 
        \newcommand*\Wrappedcontinuationindent {3ex } 
        \newcommand*\Wrappedafterbreak {\kern\Wrappedcontinuationindent\copy\Wrappedcontinuationbox} 
        % Take advantage of the already applied Pygments mark-up to insert 
        % potential linebreaks for TeX processing. 
        %        {, <, #, %, $, ' and ": go to next line. 
        %        _, }, ^, &, >, - and ~: stay at end of broken line. 
        % Use of \textquotesingle for straight quote. 
        \newcommand*\Wrappedbreaksatspecials {% 
            \def\PYGZus{\discretionary{\char`\_}{\Wrappedafterbreak}{\char`\_}}% 
            \def\PYGZob{\discretionary{}{\Wrappedafterbreak\char`\{}{\char`\{}}% 
            \def\PYGZcb{\discretionary{\char`\}}{\Wrappedafterbreak}{\char`\}}}% 
            \def\PYGZca{\discretionary{\char`\^}{\Wrappedafterbreak}{\char`\^}}% 
            \def\PYGZam{\discretionary{\char`\&}{\Wrappedafterbreak}{\char`\&}}% 
            \def\PYGZlt{\discretionary{}{\Wrappedafterbreak\char`\<}{\char`\<}}% 
            \def\PYGZgt{\discretionary{\char`\>}{\Wrappedafterbreak}{\char`\>}}% 
            \def\PYGZsh{\discretionary{}{\Wrappedafterbreak\char`\#}{\char`\#}}% 
            \def\PYGZpc{\discretionary{}{\Wrappedafterbreak\char`\%}{\char`\%}}% 
            \def\PYGZdl{\discretionary{}{\Wrappedafterbreak\char`\$}{\char`\$}}% 
            \def\PYGZhy{\discretionary{\char`\-}{\Wrappedafterbreak}{\char`\-}}% 
            \def\PYGZsq{\discretionary{}{\Wrappedafterbreak\textquotesingle}{\textquotesingle}}% 
            \def\PYGZdq{\discretionary{}{\Wrappedafterbreak\char`\"}{\char`\"}}% 
            \def\PYGZti{\discretionary{\char`\~}{\Wrappedafterbreak}{\char`\~}}% 
        } 
        % Some characters . , ; ? ! / are not pygmentized. 
        % This macro makes them "active" and they will insert potential linebreaks 
        \newcommand*\Wrappedbreaksatpunct {% 
            \lccode`\~`\.\lowercase{\def~}{\discretionary{\hbox{\char`\.}}{\Wrappedafterbreak}{\hbox{\char`\.}}}% 
            \lccode`\~`\,\lowercase{\def~}{\discretionary{\hbox{\char`\,}}{\Wrappedafterbreak}{\hbox{\char`\,}}}% 
            \lccode`\~`\;\lowercase{\def~}{\discretionary{\hbox{\char`\;}}{\Wrappedafterbreak}{\hbox{\char`\;}}}% 
            \lccode`\~`\:\lowercase{\def~}{\discretionary{\hbox{\char`\:}}{\Wrappedafterbreak}{\hbox{\char`\:}}}% 
            \lccode`\~`\?\lowercase{\def~}{\discretionary{\hbox{\char`\?}}{\Wrappedafterbreak}{\hbox{\char`\?}}}% 
            \lccode`\~`\!\lowercase{\def~}{\discretionary{\hbox{\char`\!}}{\Wrappedafterbreak}{\hbox{\char`\!}}}% 
            \lccode`\~`\/\lowercase{\def~}{\discretionary{\hbox{\char`\/}}{\Wrappedafterbreak}{\hbox{\char`\/}}}% 
            \catcode`\.\active
            \catcode`\,\active 
            \catcode`\;\active
            \catcode`\:\active
            \catcode`\?\active
            \catcode`\!\active
            \catcode`\/\active 
            \lccode`\~`\~ 	
        }
    \makeatother

    \let\OriginalVerbatim=\Verbatim
    \makeatletter
    \renewcommand{\Verbatim}[1][1]{%
        %\parskip\z@skip
        \sbox\Wrappedcontinuationbox {\Wrappedcontinuationsymbol}%
        \sbox\Wrappedvisiblespacebox {\FV@SetupFont\Wrappedvisiblespace}%
        \def\FancyVerbFormatLine ##1{\hsize\linewidth
            \vtop{\raggedright\hyphenpenalty\z@\exhyphenpenalty\z@
                \doublehyphendemerits\z@\finalhyphendemerits\z@
                \strut ##1\strut}%
        }%
        % If the linebreak is at a space, the latter will be displayed as visible
        % space at end of first line, and a continuation symbol starts next line.
        % Stretch/shrink are however usually zero for typewriter font.
        \def\FV@Space {%
            \nobreak\hskip\z@ plus\fontdimen3\font minus\fontdimen4\font
            \discretionary{\copy\Wrappedvisiblespacebox}{\Wrappedafterbreak}
            {\kern\fontdimen2\font}%
        }%
        
        % Allow breaks at special characters using \PYG... macros.
        \Wrappedbreaksatspecials
        % Breaks at punctuation characters . , ; ? ! and / need catcode=\active 	
        \OriginalVerbatim[#1,codes*=\Wrappedbreaksatpunct]%
    }
    \makeatother

    % Exact colors from NB
    \definecolor{incolor}{HTML}{303F9F}
    \definecolor{outcolor}{HTML}{D84315}
    \definecolor{cellborder}{HTML}{CFCFCF}
    \definecolor{cellbackground}{HTML}{F7F7F7}
    
    % prompt
    \newcommand{\prompt}[4]{
        \llap{{\color{#2}[#3]: #4}}\vspace{-1.25em}
    }
    

    
    % Prevent overflowing lines due to hard-to-break entities
    \sloppy 
    % Setup hyperref package
    \hypersetup{
      breaklinks=true,  % so long urls are correctly broken across lines
      colorlinks=true,
      urlcolor=urlcolor,
      linkcolor=linkcolor,
      citecolor=citecolor,
      }
    % Slightly bigger margins than the latex defaults
    
    \geometry{verbose,tmargin=1in,bmargin=1in,lmargin=1in,rmargin=1in}
    
    

    \begin{document}
    
    
    \maketitle
    
    

    
    \subsection{Newton's Second Law of
Motion}\label{newtons-second-law-of-motion}

\(y(t) = v_0 t - \frac{1}{2} g t^2\)

    \begin{tcolorbox}[breakable, size=fbox, boxrule=1pt, pad at break*=1mm,colback=cellbackground, colframe=cellborder]
\prompt{In}{incolor}{74}{\hspace{4pt}}
\begin{Verbatim}[commandchars=\\\{\}]
\PY{l+m+mi}{5} \PY{o}{*} \PY{l+m+mf}{0.6} \PY{o}{\PYZhy{}} \PY{l+m+mf}{0.5} \PY{o}{*} \PY{l+m+mf}{9.81} \PY{o}{*} \PY{l+m+mf}{0.6} \PY{o}{*}\PY{o}{*} \PY{l+m+mi}{2}
\end{Verbatim}
\end{tcolorbox}

            \begin{tcolorbox}[breakable, boxrule=.5pt, size=fbox, pad at break*=1mm, opacityfill=0]
\prompt{Out}{outcolor}{74}{\hspace{3.5pt}}
\begin{Verbatim}[commandchars=\\\{\}]
1.2342
\end{Verbatim}
\end{tcolorbox}
        
    \subsection{Height of an object}\label{height-of-an-object}

\(y(t)=v_0t-\frac{1}{2}gt^2\)

\begin{itemize}
\tightlist
\item
  v0 as initial velocity of objects
\item
  g acceleration of gravity
\item
  t as time
\end{itemize}

With y=0 as axis of object start when t=0 at initial time.

\(v_0t-\frac{1}{2}gt^2 = t(v_0-\frac{1}{2}gt)=0 \Rightarrow t=0\) or
\(t=\frac{v_0}{g}\)

\begin{itemize}
\tightlist
\item
  time to move up and return to y=0, return seconds is
  \(\frac{2 v_0}{g}\) and restricted to
  \(t \in \left[ 0, \  \frac{2 v_{0}}{g}\right]\)
\end{itemize}

    \begin{tcolorbox}[breakable, size=fbox, boxrule=1pt, pad at break*=1mm,colback=cellbackground, colframe=cellborder]
\prompt{In}{incolor}{75}{\hspace{4pt}}
\begin{Verbatim}[commandchars=\\\{\}]
\PY{c+c1}{\PYZsh{} variables for newton\PYZsq{}s second law of motion}
\PY{n}{v0} \PY{o}{=} \PY{l+m+mi}{5}
\PY{n}{g} \PY{o}{=} \PY{l+m+mf}{9.81}
\PY{n}{t} \PY{o}{=} \PY{l+m+mf}{0.6}
\PY{n}{y} \PY{o}{=} \PY{n}{v0}\PY{o}{*}\PY{n}{t} \PY{o}{\PYZhy{}} \PY{l+m+mf}{0.5}\PY{o}{*}\PY{n}{g}\PY{o}{*}\PY{n}{t}\PY{o}{*}\PY{o}{*}\PY{l+m+mi}{2}
\PY{n+nb}{print}\PY{p}{(}\PY{n}{y}\PY{p}{)}
\end{Verbatim}
\end{tcolorbox}

    \begin{Verbatim}[commandchars=\\\{\}]
1.2342
\end{Verbatim}

    \begin{tcolorbox}[breakable, size=fbox, boxrule=1pt, pad at break*=1mm,colback=cellbackground, colframe=cellborder]
\prompt{In}{incolor}{76}{\hspace{4pt}}
\begin{Verbatim}[commandchars=\\\{\}]
\PY{c+c1}{\PYZsh{} or using good pythonic naming conventions}
\PY{n}{initial\PYZus{}velocity} \PY{o}{=} \PY{l+m+mi}{5}
\PY{n}{acceleration\PYZus{}of\PYZus{}gravity} \PY{o}{=} \PY{l+m+mf}{9.81}
\PY{n}{TIME} \PY{o}{=} \PY{l+m+mf}{0.6}
\PY{n}{VerticalPositionOfBall} \PY{o}{=} \PY{n}{initial\PYZus{}velocity}\PY{o}{*}\PY{n}{TIME} \PY{o}{\PYZhy{}} \PYZbs{}
                         \PY{l+m+mf}{0.5}\PY{o}{*}\PY{n}{acceleration\PYZus{}of\PYZus{}gravity}\PY{o}{*}\PY{n}{TIME}\PY{o}{*}\PY{o}{*}\PY{l+m+mi}{2}
\PY{n+nb}{print}\PY{p}{(}\PY{n}{VerticalPositionOfBall}\PY{p}{)}
\end{Verbatim}
\end{tcolorbox}

    \begin{Verbatim}[commandchars=\\\{\}]
1.2342
\end{Verbatim}

    \subsection{Integral calculation}\label{integral-calculation}

\[
\int_{-\infty}^1 e^{-x^2}dx{\thinspace .}
\]

    \begin{tcolorbox}[breakable, size=fbox, boxrule=1pt, pad at break*=1mm,colback=cellbackground, colframe=cellborder]
\prompt{In}{incolor}{77}{\hspace{4pt}}
\begin{Verbatim}[commandchars=\\\{\}]
\PY{k+kn}{from} \PY{n+nn}{numpy} \PY{k}{import} \PY{o}{*}

\PY{k}{def} \PY{n+nf}{integrate}\PY{p}{(}\PY{n}{f}\PY{p}{,} \PY{n}{a}\PY{p}{,} \PY{n}{b}\PY{p}{,} \PY{n}{n}\PY{o}{=}\PY{l+m+mi}{100}\PY{p}{)}\PY{p}{:}
    \PY{l+s+sd}{\PYZdq{}\PYZdq{}\PYZdq{}}
\PY{l+s+sd}{    Integrate f from a to b}
\PY{l+s+sd}{    using the Trapezoildal rule with n intervals.}
\PY{l+s+sd}{    \PYZdq{}\PYZdq{}\PYZdq{}}
    \PY{n}{x} \PY{o}{=} \PY{n}{linspace}\PY{p}{(}\PY{n}{a}\PY{p}{,} \PY{n}{b}\PY{p}{,} \PY{n}{n}\PY{o}{+}\PY{l+m+mi}{1}\PY{p}{)}  \PY{c+c1}{\PYZsh{} coords of intervals}
    \PY{n}{h} \PY{o}{=} \PY{n}{x}\PY{p}{[}\PY{l+m+mi}{1}\PY{p}{]} \PY{o}{\PYZhy{}} \PY{n}{x}\PY{p}{[}\PY{l+m+mi}{0}\PY{p}{]}
    \PY{n}{I} \PY{o}{=} \PY{n}{h}\PY{o}{*}\PY{p}{(}\PY{n+nb}{sum}\PY{p}{(}\PY{n}{f}\PY{p}{(}\PY{n}{x}\PY{p}{)}\PY{p}{)} \PY{o}{\PYZhy{}} \PY{l+m+mf}{0.5}\PY{o}{*}\PY{p}{(}\PY{n}{f}\PY{p}{(}\PY{n}{a}\PY{p}{)} \PY{o}{+} \PY{n}{f}\PY{p}{(}\PY{n}{b}\PY{p}{)}\PY{p}{)}\PY{p}{)}
    \PY{k}{return} \PY{n}{I}

\PY{c+c1}{\PYZsh{} define integrand}
\PY{k}{def} \PY{n+nf}{my\PYZus{}function}\PY{p}{(}\PY{n}{x}\PY{p}{)}\PY{p}{:}
    \PY{k}{return} \PY{n}{exp}\PY{p}{(}\PY{o}{\PYZhy{}}\PY{n}{x}\PY{o}{*}\PY{o}{*}\PY{l+m+mi}{2}\PY{p}{)}

\PY{n}{minus\PYZus{}infinity} \PY{o}{=} \PY{o}{\PYZhy{}}\PY{l+m+mi}{20} \PY{c+c1}{\PYZsh{} aprox for minus infinity}
\PY{n}{I} \PY{o}{=} \PY{n}{integrate}\PY{p}{(}\PY{n}{my\PYZus{}function}\PY{p}{,} \PY{n}{minus\PYZus{}infinity}\PY{p}{,} \PY{l+m+mi}{1}\PY{p}{,} \PY{n}{n}\PY{o}{=}\PY{l+m+mi}{1000}\PY{p}{)}
\PY{n+nb}{print}\PY{p}{(}\PY{l+s+s2}{\PYZdq{}}\PY{l+s+s2}{value of integral:}\PY{l+s+s2}{\PYZdq{}}\PY{p}{,} \PY{n}{I}\PY{p}{)}
\end{Verbatim}
\end{tcolorbox}

    \begin{Verbatim}[commandchars=\\\{\}]
value of integral: 1.6330240187288536
\end{Verbatim}

    \begin{tcolorbox}[breakable, size=fbox, boxrule=1pt, pad at break*=1mm,colback=cellbackground, colframe=cellborder]
\prompt{In}{incolor}{78}{\hspace{4pt}}
\begin{Verbatim}[commandchars=\\\{\}]
\PY{c+c1}{\PYZsh{} Celsius\PYZhy{}Fahrenheit Conversion}
\PY{n}{C} \PY{o}{=} \PY{l+m+mi}{21}
\PY{n}{F} \PY{o}{=} \PY{p}{(}\PY{l+m+mi}{9}\PY{o}{/}\PY{l+m+mi}{5}\PY{p}{)}\PY{o}{*}\PY{n}{C} \PY{o}{+} \PY{l+m+mi}{32}
\PY{n+nb}{print}\PY{p}{(}\PY{n}{F}\PY{p}{)}
\end{Verbatim}
\end{tcolorbox}

    \begin{Verbatim}[commandchars=\\\{\}]
69.80000000000001
\end{Verbatim}

    \subsection{\texorpdfstring{Time to reach height of
\(y_c\)}{Time to reach height of y\_c}}\label{time-to-reach-height-of-y_c}

\(y_c =v_0 t - \frac{1}{2} g t^2\)

Quadratic equation to solve.

\(\frac{1}{2}gt^2-v_0t+y_c=0\)

\(t_1=\Bigg(v_0-\sqrt{v_0^2-2gy_c}\Bigg)/g\quad\)up\(\quad(t=t_1)\)

\(t_2=\Bigg(v_0+\sqrt{v_0^2-2gy_c}\Bigg)/g\quad\)down\(\quad(t=t_2>t_1)\)

    \begin{tcolorbox}[breakable, size=fbox, boxrule=1pt, pad at break*=1mm,colback=cellbackground, colframe=cellborder]
\prompt{In}{incolor}{79}{\hspace{4pt}}
\begin{Verbatim}[commandchars=\\\{\}]
\PY{n}{v0} \PY{o}{=} \PY{l+m+mi}{5}
\PY{n}{g} \PY{o}{=} \PY{l+m+mf}{9.81}
\PY{n}{yc} \PY{o}{=} \PY{l+m+mf}{0.2}
\PY{k+kn}{import} \PY{n+nn}{math}
\PY{n}{t1} \PY{o}{=} \PY{p}{(}\PY{n}{v0} \PY{o}{\PYZhy{}} \PY{n}{math}\PY{o}{.}\PY{n}{sqrt}\PY{p}{(}\PY{n}{v0}\PY{o}{*}\PY{o}{*}\PY{l+m+mi}{2} \PY{o}{\PYZhy{}} \PY{l+m+mi}{2} \PY{o}{*} \PY{n}{g} \PY{o}{*} \PY{n}{yc}\PY{p}{)}\PY{p}{)} \PY{o}{/} \PY{n}{g}
\PY{n}{t2} \PY{o}{=} \PY{p}{(}\PY{n}{v0} \PY{o}{+} \PY{n}{math}\PY{o}{.}\PY{n}{sqrt}\PY{p}{(}\PY{n}{v0}\PY{o}{*}\PY{o}{*}\PY{l+m+mi}{2} \PY{o}{\PYZhy{}} \PY{l+m+mi}{2} \PY{o}{*} \PY{n}{g} \PY{o}{*} \PY{n}{yc}\PY{p}{)}\PY{p}{)} \PY{o}{/} \PY{n}{g}
\PY{n+nb}{print}\PY{p}{(}\PY{l+s+s1}{\PYZsq{}}\PY{l+s+s1}{At t=}\PY{l+s+si}{\PYZpc{}g}\PY{l+s+s1}{ s and }\PY{l+s+si}{\PYZpc{}g}\PY{l+s+s1}{ s, the height is }\PY{l+s+si}{\PYZpc{}g}\PY{l+s+s1}{ m.}\PY{l+s+s1}{\PYZsq{}} \PY{o}{\PYZpc{}} \PY{p}{(}\PY{n}{t1}\PY{p}{,} \PY{n}{t2}\PY{p}{,} \PY{n}{yc}\PY{p}{)}\PY{p}{)}
\end{Verbatim}
\end{tcolorbox}

    \begin{Verbatim}[commandchars=\\\{\}]
At t=0.0417064 s and 0.977662 s, the height is 0.2 m.
\end{Verbatim}

    \subsection{\texorpdfstring{The hyperbolic sine function
\(sinh(x) = \frac{1}{2}(e^x - e^{-x})\) and other math functions with
right hand
sides.}{The hyperbolic sine function sinh(x) = \textbackslash{}frac\{1\}\{2\}(e\^{}x - e\^{}\{-x\}) and other math functions with right hand sides.}}\label{the-hyperbolic-sine-function-sinhx-frac12ex---e-x-and-other-math-functions-with-right-hand-sides.}

    \begin{tcolorbox}[breakable, size=fbox, boxrule=1pt, pad at break*=1mm,colback=cellbackground, colframe=cellborder]
\prompt{In}{incolor}{80}{\hspace{4pt}}
\begin{Verbatim}[commandchars=\\\{\}]
\PY{k+kn}{from} \PY{n+nn}{math} \PY{k}{import} \PY{n}{sinh}\PY{p}{,} \PY{n}{exp}\PY{p}{,} \PY{n}{e}\PY{p}{,} \PY{n}{pi}
\PY{n}{x} \PY{o}{=} \PY{l+m+mi}{2}\PY{o}{*}\PY{n}{pi}
\PY{n}{r1} \PY{o}{=} \PY{n}{sinh}\PY{p}{(}\PY{n}{x}\PY{p}{)}
\PY{n}{r2} \PY{o}{=} \PY{l+m+mf}{0.5}\PY{o}{*}\PY{p}{(}\PY{n}{exp}\PY{p}{(}\PY{n}{x}\PY{p}{)} \PY{o}{\PYZhy{}} \PY{n}{exp}\PY{p}{(}\PY{o}{\PYZhy{}}\PY{n}{x}\PY{p}{)}\PY{p}{)}
\PY{n}{r3} \PY{o}{=} \PY{l+m+mf}{0.5}\PY{o}{*}\PY{p}{(}\PY{n}{e}\PY{o}{*}\PY{o}{*}\PY{n}{x} \PY{o}{\PYZhy{}} \PY{n}{e}\PY{o}{*}\PY{o}{*}\PY{p}{(}\PY{o}{\PYZhy{}}\PY{n}{x}\PY{p}{)}\PY{p}{)}
\PY{n+nb}{print}\PY{p}{(}\PY{n}{r1}\PY{p}{,} \PY{n}{r2}\PY{p}{,} \PY{n}{r3}\PY{p}{)} \PY{c+c1}{\PYZsh{} with rounding errors}
\end{Verbatim}
\end{tcolorbox}

    \begin{Verbatim}[commandchars=\\\{\}]
267.74489404101644 267.74489404101644 267.7448940410163
\end{Verbatim}

    \begin{tcolorbox}[breakable, size=fbox, boxrule=1pt, pad at break*=1mm,colback=cellbackground, colframe=cellborder]
\prompt{In}{incolor}{81}{\hspace{4pt}}
\begin{Verbatim}[commandchars=\\\{\}]
\PY{c+c1}{\PYZsh{} Math functions for complex numbers}
\PY{k+kn}{from} \PY{n+nn}{scipy} \PY{k}{import} \PY{o}{*}

\PY{k+kn}{from} \PY{n+nn}{cmath} \PY{k}{import} \PY{n}{sqrt}
\PY{n}{sqrt}\PY{p}{(}\PY{o}{\PYZhy{}}\PY{l+m+mi}{1}\PY{p}{)}  \PY{c+c1}{\PYZsh{} complex number with cmath}

\PY{k+kn}{from} \PY{n+nn}{numpy}\PY{n+nn}{.}\PY{n+nn}{lib}\PY{n+nn}{.}\PY{n+nn}{scimath} \PY{k}{import} \PY{n}{sqrt}
\PY{n}{a} \PY{o}{=} \PY{l+m+mi}{1}\PY{p}{;} \PY{n}{b} \PY{o}{=} \PY{l+m+mi}{2}\PY{p}{;} \PY{n}{c} \PY{o}{=} \PY{l+m+mi}{100}
\PY{n}{r1} \PY{o}{=} \PY{p}{(}\PY{o}{\PYZhy{}}\PY{n}{b} \PY{o}{+} \PY{n}{sqrt}\PY{p}{(}\PY{n}{b}\PY{o}{*}\PY{o}{*}\PY{l+m+mi}{2} \PY{o}{\PYZhy{}} \PY{l+m+mi}{4}\PY{o}{*}\PY{n}{a}\PY{o}{*}\PY{n}{c}\PY{p}{)}\PY{p}{)}\PY{o}{/}\PY{p}{(}\PY{l+m+mi}{2}\PY{o}{*}\PY{n}{a}\PY{p}{)}
\PY{n}{r2} \PY{o}{=} \PY{p}{(}\PY{o}{\PYZhy{}}\PY{n}{b} \PY{o}{\PYZhy{}} \PY{n}{sqrt}\PY{p}{(}\PY{n}{b}\PY{o}{*}\PY{o}{*}\PY{l+m+mi}{2} \PY{o}{\PYZhy{}} \PY{l+m+mi}{4}\PY{o}{*}\PY{n}{a}\PY{o}{*}\PY{n}{c}\PY{p}{)}\PY{p}{)}\PY{o}{/}\PY{p}{(}\PY{l+m+mi}{2}\PY{o}{*}\PY{n}{a}\PY{p}{)}
\PY{n+nb}{print}\PY{p}{(}\PY{l+s+s2}{\PYZdq{}\PYZdq{}\PYZdq{}}
\PY{l+s+s2}{t1=}\PY{l+s+si}{\PYZob{}r1:g\PYZcb{}}
\PY{l+s+s2}{t2=}\PY{l+s+si}{\PYZob{}r2:g\PYZcb{}}\PY{l+s+s2}{\PYZdq{}\PYZdq{}\PYZdq{}}\PY{o}{.}\PY{n}{format}\PY{p}{(}\PY{n}{r1}\PY{o}{=}\PY{n}{r1}\PY{p}{,} \PY{n}{r2}\PY{o}{=}\PY{n}{r2}\PY{p}{)}\PY{p}{)}
\end{Verbatim}
\end{tcolorbox}

    \begin{Verbatim}[commandchars=\\\{\}]

t1=-1+9.94987j
t2=-1-9.94987j
\end{Verbatim}

    \begin{tcolorbox}[breakable, size=fbox, boxrule=1pt, pad at break*=1mm,colback=cellbackground, colframe=cellborder]
\prompt{In}{incolor}{82}{\hspace{4pt}}
\begin{Verbatim}[commandchars=\\\{\}]
\PY{c+c1}{\PYZsh{} Symbolic computing}
\PY{k+kn}{from} \PY{n+nn}{sympy} \PY{k}{import} \PY{p}{(}
    \PY{n}{symbols}\PY{p}{,}  \PY{c+c1}{\PYZsh{} define symbols for symbolic math}
    \PY{n}{diff}\PY{p}{,}  \PY{c+c1}{\PYZsh{} differentiate expressions}
    \PY{n}{integrate}\PY{p}{,}  \PY{c+c1}{\PYZsh{} integrate expressions}
    \PY{n}{Rational}\PY{p}{,}  \PY{c+c1}{\PYZsh{} define rational numbers}
    \PY{n}{lambdify}\PY{p}{,}  \PY{c+c1}{\PYZsh{} turn symbolic expr. into python functions}
    \PY{p}{)}

\PY{c+c1}{\PYZsh{} declare symbolic variables}
\PY{n}{t}\PY{p}{,} \PY{n}{v0}\PY{p}{,} \PY{n}{g} \PY{o}{=} \PY{n}{symbols}\PY{p}{(}\PY{l+s+s1}{\PYZsq{}}\PY{l+s+s1}{t v0 g}\PY{l+s+s1}{\PYZsq{}}\PY{p}{)}
\PY{c+c1}{\PYZsh{} formula}
\PY{n}{y} \PY{o}{=} \PY{n}{v0}\PY{o}{*}\PY{n}{t} \PY{o}{\PYZhy{}} \PY{n}{Rational}\PY{p}{(}\PY{l+m+mi}{1}\PY{p}{,}\PY{l+m+mi}{2}\PY{p}{)}\PY{o}{*}\PY{n}{g}\PY{o}{*}\PY{n}{t}\PY{o}{*}\PY{o}{*}\PY{l+m+mi}{2}
\PY{n}{dydt} \PY{o}{=} \PY{n}{diff}\PY{p}{(}\PY{n}{y} \PY{p}{,}\PY{n}{t}\PY{p}{)}
\PY{n+nb}{print}\PY{p}{(}\PY{l+s+s2}{\PYZdq{}}\PY{l+s+s2}{At time}\PY{l+s+s2}{\PYZdq{}}\PY{p}{,} \PY{n}{dydt}\PY{p}{)}
\PY{n+nb}{print}\PY{p}{(}\PY{l+s+s2}{\PYZdq{}}\PY{l+s+s2}{acceleration:}\PY{l+s+s2}{\PYZdq{}}\PY{p}{,} \PY{n}{diff}\PY{p}{(}\PY{n}{y}\PY{p}{,}\PY{n}{t}\PY{p}{,}\PY{n}{t}\PY{p}{)}\PY{p}{)} \PY{c+c1}{\PYZsh{} 2nd derivative}
\PY{n}{y2} \PY{o}{=} \PY{n}{integrate}\PY{p}{(}\PY{n}{dydt}\PY{p}{,} \PY{n}{t}\PY{p}{)}
\PY{n+nb}{print}\PY{p}{(}\PY{l+s+s2}{\PYZdq{}}\PY{l+s+s2}{integration of dydt wrt t}\PY{l+s+s2}{\PYZdq{}}\PY{p}{,} \PY{n}{y2}\PY{p}{)}

\PY{c+c1}{\PYZsh{} convert to python function}
\PY{n}{v} \PY{o}{=} \PY{n}{lambdify}\PY{p}{(}\PY{p}{[}\PY{n}{t}\PY{p}{,} \PY{n}{v0}\PY{p}{,} \PY{n}{g}\PY{p}{]}\PY{p}{,}  \PY{c+c1}{\PYZsh{} arguments in v}
             \PY{n}{dydt}\PY{p}{)}  \PY{c+c1}{\PYZsh{} symbolic expression}
\PY{n+nb}{print}\PY{p}{(}\PY{l+s+s2}{\PYZdq{}}\PY{l+s+s2}{As a function compute y = }\PY{l+s+si}{\PYZpc{}g}\PY{l+s+s2}{\PYZdq{}} \PY{o}{\PYZpc{}} \PY{n}{v}\PY{p}{(}\PY{n}{t}\PY{o}{=}\PY{l+m+mi}{0}\PY{p}{,} \PY{n}{v0}\PY{o}{=}\PY{l+m+mi}{5}\PY{p}{,} \PY{n}{g}\PY{o}{=}\PY{l+m+mf}{9.81}\PY{p}{)}\PY{p}{)}
\end{Verbatim}
\end{tcolorbox}

    \begin{Verbatim}[commandchars=\\\{\}]
At time -g*t + v0
acceleration: -g
integration of dydt wrt t -g*t**2/2 + t*v0
As a function compute y = 5
\end{Verbatim}

    \begin{tcolorbox}[breakable, size=fbox, boxrule=1pt, pad at break*=1mm,colback=cellbackground, colframe=cellborder]
\prompt{In}{incolor}{83}{\hspace{4pt}}
\begin{Verbatim}[commandchars=\\\{\}]
\PY{c+c1}{\PYZsh{} equation solving for expression e=0, t unknown}
\PY{k+kn}{from} \PY{n+nn}{sympy} \PY{k}{import} \PY{n}{solve}
\PY{n}{roots} \PY{o}{=} \PY{n}{solve}\PY{p}{(}\PY{n}{y}\PY{p}{,} \PY{n}{t}\PY{p}{)}  \PY{c+c1}{\PYZsh{} e is y}
\PY{n+nb}{print}\PY{p}{(}\PY{l+s+s2}{\PYZdq{}\PYZdq{}\PYZdq{}}
\PY{l+s+s2}{If y = 0 for t then t solves y for [}\PY{l+s+si}{\PYZob{}\PYZcb{}}\PY{l+s+s2}{,}\PY{l+s+si}{\PYZob{}\PYZcb{}}\PY{l+s+s2}{].}

\PY{l+s+s2}{\PYZdq{}\PYZdq{}\PYZdq{}}\PY{o}{.}\PY{n}{format}\PY{p}{(}
            \PY{n}{y}\PY{o}{.}\PY{n}{subs}\PY{p}{(}\PY{n}{t}\PY{p}{,} \PY{n}{roots}\PY{p}{[}\PY{l+m+mi}{0}\PY{p}{]}\PY{p}{)}\PY{p}{,}
            \PY{n}{y}\PY{o}{.}\PY{n}{subs}\PY{p}{(}\PY{n}{t}\PY{p}{,} \PY{n}{roots}\PY{p}{[}\PY{l+m+mi}{1}\PY{p}{]}\PY{p}{)}
          \PY{p}{)} \PY{p}{)}
\end{Verbatim}
\end{tcolorbox}

    \begin{Verbatim}[commandchars=\\\{\}]

If y = 0 for t then t solves y for [0,0].


\end{Verbatim}

    \(y(t)=v_0t-\frac{1}{2}gt^2\), \(t \in [0, \frac{2 v_0}{g}]\)

    \begin{tcolorbox}[breakable, size=fbox, boxrule=1pt, pad at break*=1mm,colback=cellbackground, colframe=cellborder]
\prompt{In}{incolor}{84}{\hspace{4pt}}
\begin{Verbatim}[commandchars=\\\{\}]
\PY{c+c1}{\PYZsh{} Taylor series to the order n in a variable t around the point t0}
\PY{k+kn}{from} \PY{n+nn}{sympy} \PY{k}{import} \PY{n}{exp}\PY{p}{,} \PY{n}{sin}\PY{p}{,} \PY{n}{cos}
\PY{n}{f} \PY{o}{=} \PY{n}{exp}\PY{p}{(}\PY{n}{t}\PY{p}{)}
\PY{n}{f}\PY{o}{.}\PY{n}{series}\PY{p}{(}\PY{n}{t}\PY{p}{,} \PY{l+m+mi}{0}\PY{p}{,} \PY{l+m+mi}{3}\PY{p}{)}
\PY{n}{f\PYZus{}sin} \PY{o}{=} \PY{n}{exp}\PY{p}{(}\PY{n}{sin}\PY{p}{(}\PY{n}{t}\PY{p}{)}\PY{p}{)}
\PY{n}{f\PYZus{}sin}\PY{o}{.}\PY{n}{series}\PY{p}{(}\PY{n}{t}\PY{p}{,} \PY{l+m+mi}{0}\PY{p}{,} \PY{l+m+mi}{8}\PY{p}{)}
\end{Verbatim}
\end{tcolorbox}
 
            
\prompt{Out}{outcolor}{84}{}
    
    $\displaystyle 1 + t + \frac{t^{2}}{2} - \frac{t^{4}}{8} - \frac{t^{5}}{15} - \frac{t^{6}}{240} + \frac{t^{7}}{90} + O\left(t^{8}\right)$

    

    \subsection{Taylor Series Polynomial to approximate
functions;}\label{taylor-series-polynomial-to-approximate-functions}

\(1 + t + \frac{t^{2}}{2} - \frac{t^{4}}{8} - \frac{t^{5}}{15} - \frac{t^{6}}{240} + \frac{t^{7}}{90} + O\left(t^{8}\right)\)

    \begin{tcolorbox}[breakable, size=fbox, boxrule=1pt, pad at break*=1mm,colback=cellbackground, colframe=cellborder]
\prompt{In}{incolor}{85}{\hspace{4pt}}
\begin{Verbatim}[commandchars=\\\{\}]
\PY{c+c1}{\PYZsh{} expanding and simplifying expressions}
\PY{k+kn}{from} \PY{n+nn}{sympy} \PY{k}{import} \PY{n}{simplify}\PY{p}{,} \PY{n}{expand}
\PY{n}{x}\PY{p}{,} \PY{n}{y} \PY{o}{=} \PY{n}{symbols}\PY{p}{(}\PY{l+s+s1}{\PYZsq{}}\PY{l+s+s1}{x y}\PY{l+s+s1}{\PYZsq{}}\PY{p}{)}
\PY{n}{f} \PY{o}{=} \PY{o}{\PYZhy{}}\PY{n}{sin}\PY{p}{(}\PY{n}{x}\PY{p}{)} \PY{o}{*} \PY{n}{sin}\PY{p}{(}\PY{n}{y}\PY{p}{)} \PY{o}{+} \PY{n}{cos}\PY{p}{(}\PY{n}{x}\PY{p}{)} \PY{o}{*} \PY{n}{cos}\PY{p}{(}\PY{n}{y}\PY{p}{)}
\PY{n+nb}{print}\PY{p}{(}\PY{n}{f}\PY{p}{)}
\PY{n+nb}{print}\PY{p}{(}\PY{n}{simplify}\PY{p}{(}\PY{n}{f}\PY{p}{)}\PY{p}{)}
\PY{n+nb}{print}\PY{p}{(}\PY{n}{expand}\PY{p}{(}\PY{n}{sin}\PY{p}{(}\PY{n}{x} \PY{o}{+} \PY{n}{y}\PY{p}{)}\PY{p}{,} \PY{n}{trig}\PY{o}{=}\PY{k+kc}{True}\PY{p}{)}\PY{p}{)}  \PY{c+c1}{\PYZsh{} expand as trig funct}
\end{Verbatim}
\end{tcolorbox}

    \begin{Verbatim}[commandchars=\\\{\}]
-sin(x)*sin(y) + cos(x)*cos(y)
cos(x + y)
sin(x)*cos(y) + sin(y)*cos(x)
\end{Verbatim}

    \subsection{Trajectory of an object}\label{trajectory-of-an-object}

\[f(x) = x tan \theta - \frac{1}{2 v^{2}_{0}} \cdot
\frac{gx^2}{cos^{2}\theta} + y_0\]

    \begin{tcolorbox}[breakable, size=fbox, boxrule=1pt, pad at break*=1mm,colback=cellbackground, colframe=cellborder]
\prompt{In}{incolor}{86}{\hspace{4pt}}
\begin{Verbatim}[commandchars=\\\{\}]
\PY{c+c1}{\PYZsh{} Trajectory of an object}
\PY{n}{g} \PY{o}{=} \PY{l+m+mf}{9.81}      \PY{c+c1}{\PYZsh{} m/s**2}
\PY{n}{v0} \PY{o}{=} \PY{l+m+mi}{15}       \PY{c+c1}{\PYZsh{} km/h}
\PY{n}{theta} \PY{o}{=} \PY{l+m+mi}{60}    \PY{c+c1}{\PYZsh{} degree}
\PY{n}{x} \PY{o}{=} \PY{l+m+mf}{0.5}       \PY{c+c1}{\PYZsh{} m}
\PY{n}{y0} \PY{o}{=} \PY{l+m+mi}{1}        \PY{c+c1}{\PYZsh{} m}

\PY{n+nb}{print}\PY{p}{(}\PY{l+s+s2}{\PYZdq{}\PYZdq{}\PYZdq{}}\PY{l+s+se}{\PYZbs{}}
\PY{l+s+s2}{v0      = }\PY{l+s+si}{\PYZpc{}.1f}\PY{l+s+s2}{ km/h}
\PY{l+s+s2}{theta   = }\PY{l+s+si}{\PYZpc{}d}\PY{l+s+s2}{ degree}
\PY{l+s+s2}{y0      = }\PY{l+s+si}{\PYZpc{}.1f}\PY{l+s+s2}{ m}
\PY{l+s+s2}{x       = }\PY{l+s+si}{\PYZpc{}.1f}\PY{l+s+s2}{ m}\PY{l+s+se}{\PYZbs{}}
\PY{l+s+s2}{\PYZdq{}\PYZdq{}\PYZdq{}} \PY{o}{\PYZpc{}} \PY{p}{(}\PY{n}{v0}\PY{p}{,} \PY{n}{theta}\PY{p}{,} \PY{n}{y0}\PY{p}{,} \PY{n}{x}\PY{p}{)}\PY{p}{)}

\PY{k+kn}{from} \PY{n+nn}{math} \PY{k}{import} \PY{n}{pi}\PY{p}{,} \PY{n}{tan}\PY{p}{,} \PY{n}{cos}
\PY{n}{v0} \PY{o}{=} \PY{n}{v0}\PY{o}{/}\PY{l+m+mf}{3.6}             \PY{c+c1}{\PYZsh{} km/h 1000/1 to m/s 1/60}
\PY{n}{theta} \PY{o}{=} \PY{n}{theta}\PY{o}{*}\PY{n}{pi}\PY{o}{/}\PY{l+m+mi}{180}    \PY{c+c1}{\PYZsh{} degree to radians}

\PY{n}{y} \PY{o}{=} \PY{n}{x}\PY{o}{*}\PY{n}{tan}\PY{p}{(}\PY{n}{theta}\PY{p}{)} \PY{o}{\PYZhy{}} \PY{l+m+mi}{1}\PY{o}{/}\PY{p}{(}\PY{l+m+mi}{2}\PY{o}{*}\PY{n}{v0}\PY{o}{*}\PY{o}{*}\PY{l+m+mi}{2}\PY{p}{)}\PY{o}{*}\PY{n}{g}\PY{o}{*}\PY{n}{x}\PY{o}{*}\PY{o}{*}\PY{l+m+mi}{2}\PY{o}{/}\PY{p}{(}\PY{p}{(}\PY{n}{cos}\PY{p}{(}\PY{n}{theta}\PY{p}{)}\PY{p}{)}\PY{o}{*}\PY{o}{*}\PY{l+m+mi}{2}\PY{p}{)}\PY{o}{+}\PY{n}{y0}
\PY{n+nb}{print}\PY{p}{(}\PY{l+s+s2}{\PYZdq{}}\PY{l+s+s2}{y       = }\PY{l+s+si}{\PYZpc{}.1f}\PY{l+s+s2}{ m}\PY{l+s+s2}{\PYZdq{}} \PY{o}{\PYZpc{}} \PY{n}{y}\PY{p}{)}
\end{Verbatim}
\end{tcolorbox}

    \begin{Verbatim}[commandchars=\\\{\}]
v0      = 15.0 km/h
theta   = 60 degree
y0      = 1.0 m
x       = 0.5 m
y       = 1.6 m
\end{Verbatim}

    \subsection{Conversion from meters to British
units}\label{conversion-from-meters-to-british-units}

    \begin{tcolorbox}[breakable, size=fbox, boxrule=1pt, pad at break*=1mm,colback=cellbackground, colframe=cellborder]
\prompt{In}{incolor}{87}{\hspace{4pt}}
\begin{Verbatim}[commandchars=\\\{\}]
\PY{c+c1}{\PYZsh{} Convert meters to british length.}
\PY{n}{meters} \PY{o}{=} \PY{l+m+mi}{640}
\PY{n}{m} \PY{o}{=} \PY{n}{symbols}\PY{p}{(}\PY{l+s+s1}{\PYZsq{}}\PY{l+s+s1}{m}\PY{l+s+s1}{\PYZsq{}}\PY{p}{)}
\PY{n}{in\PYZus{}m} \PY{o}{=} \PY{n}{m}\PY{o}{/}\PY{p}{(}\PY{l+m+mf}{2.54}\PY{p}{)}\PY{o}{*}\PY{l+m+mi}{100}
\PY{n}{ft\PYZus{}m} \PY{o}{=} \PY{n}{in\PYZus{}m} \PY{o}{/} \PY{l+m+mi}{12}
\PY{n}{yrd\PYZus{}m} \PY{o}{=} \PY{n}{ft\PYZus{}m} \PY{o}{/} \PY{l+m+mi}{3}
\PY{n}{bm\PYZus{}m} \PY{o}{=} \PY{n}{yrd\PYZus{}m} \PY{o}{/} \PY{l+m+mi}{1760}

\PY{n}{f\PYZus{}in\PYZus{}m} \PY{o}{=} \PY{n}{lambdify}\PY{p}{(}\PY{p}{[}\PY{n}{m}\PY{p}{]}\PY{p}{,} \PY{n}{in\PYZus{}m}\PY{p}{)}
\PY{n}{f\PYZus{}ft\PYZus{}m} \PY{o}{=} \PY{n}{lambdify}\PY{p}{(}\PY{p}{[}\PY{n}{m}\PY{p}{]}\PY{p}{,} \PY{n}{ft\PYZus{}m}\PY{p}{)}
\PY{n}{f\PYZus{}yrd\PYZus{}m} \PY{o}{=} \PY{n}{lambdify}\PY{p}{(}\PY{p}{[}\PY{n}{m}\PY{p}{]}\PY{p}{,} \PY{n}{yrd\PYZus{}m}\PY{p}{)}
\PY{n}{f\PYZus{}bm\PYZus{}m} \PY{o}{=} \PY{n}{lambdify}\PY{p}{(}\PY{p}{[}\PY{n}{m}\PY{p}{]}\PY{p}{,} \PY{n}{bm\PYZus{}m}\PY{p}{)}

\PY{n+nb}{print}\PY{p}{(}\PY{l+s+s2}{\PYZdq{}\PYZdq{}\PYZdq{}}
\PY{l+s+s2}{Given }\PY{l+s+si}{\PYZob{}meters:g\PYZcb{}}\PY{l+s+s2}{ meters conversions for;}
\PY{l+s+s2}{inches are }\PY{l+s+si}{\PYZob{}inches:.2f\PYZcb{}}\PY{l+s+s2}{ in}
\PY{l+s+s2}{feet are }\PY{l+s+si}{\PYZob{}feet:.2f\PYZcb{}}\PY{l+s+s2}{ ft}
\PY{l+s+s2}{yards are }\PY{l+s+si}{\PYZob{}yards:.2f\PYZcb{}}\PY{l+s+s2}{ yd}
\PY{l+s+s2}{miles are }\PY{l+s+si}{\PYZob{}miles:.3f\PYZcb{}}\PY{l+s+s2}{ m}
\PY{l+s+s2}{\PYZdq{}\PYZdq{}\PYZdq{}}\PY{o}{.}\PY{n}{format}\PY{p}{(}\PY{n}{meters}\PY{o}{=}\PY{n}{meters}\PY{p}{,}
           \PY{n}{inches}\PY{o}{=}\PY{n}{f\PYZus{}in\PYZus{}m}\PY{p}{(}\PY{n}{meters}\PY{p}{)}\PY{p}{,}
           \PY{n}{feet}\PY{o}{=}\PY{n}{f\PYZus{}ft\PYZus{}m}\PY{p}{(}\PY{n}{meters}\PY{p}{)}\PY{p}{,}
           \PY{n}{yards}\PY{o}{=}\PY{n}{f\PYZus{}yrd\PYZus{}m}\PY{p}{(}\PY{n}{meters}\PY{p}{)}\PY{p}{,}
           \PY{n}{miles}\PY{o}{=}\PY{n}{f\PYZus{}bm\PYZus{}m}\PY{p}{(}\PY{n}{meters}\PY{p}{)}\PY{p}{)}\PY{p}{)}
\end{Verbatim}
\end{tcolorbox}

    \begin{Verbatim}[commandchars=\\\{\}]

Given 640 meters conversions for;
inches are 25196.85 in
feet are 2099.74 ft
yards are 699.91 yd
miles are 0.398 m

\end{Verbatim}

    \subsection{Gaussian function}\label{gaussian-function}

\[f(x) = \frac{1}{\sqrt{2\pi}s} \text{exp} \Bigg[
-{\frac{1}{2} \Big( \frac{x-m}{s} \Big)^2} \Bigg]\]

    \begin{tcolorbox}[breakable, size=fbox, boxrule=1pt, pad at break*=1mm,colback=cellbackground, colframe=cellborder]
\prompt{In}{incolor}{88}{\hspace{4pt}}
\begin{Verbatim}[commandchars=\\\{\}]
\PY{k+kn}{from} \PY{n+nn}{sympy} \PY{k}{import} \PY{n}{pi}\PY{p}{,} \PY{n}{exp}\PY{p}{,} \PY{n}{sqrt}\PY{p}{,} \PY{n}{symbols}\PY{p}{,} \PY{n}{lambdify}

\PY{n}{s}\PY{p}{,} \PY{n}{x}\PY{p}{,} \PY{n}{m} \PY{o}{=} \PY{n}{symbols}\PY{p}{(}\PY{l+s+s2}{\PYZdq{}}\PY{l+s+s2}{s x m}\PY{l+s+s2}{\PYZdq{}}\PY{p}{)}
\PY{n}{y} \PY{o}{=} \PY{l+m+mi}{1}\PY{o}{/} \PY{p}{(}\PY{n}{sqrt}\PY{p}{(}\PY{l+m+mi}{2}\PY{o}{*}\PY{n}{pi}\PY{p}{)}\PY{o}{*}\PY{n}{s}\PY{p}{)} \PY{o}{*} \PY{n}{exp}\PY{p}{(}\PY{o}{\PYZhy{}}\PY{l+m+mf}{0.5}\PY{o}{*}\PY{p}{(}\PY{p}{(}\PY{n}{x}\PY{o}{\PYZhy{}}\PY{n}{m}\PY{p}{)}\PY{o}{/}\PY{n}{s}\PY{p}{)}\PY{o}{*}\PY{o}{*}\PY{l+m+mi}{2}\PY{p}{)}
\PY{n}{gaus\PYZus{}d} \PY{o}{=} \PY{n}{lambdify}\PY{p}{(}\PY{p}{[}\PY{n}{m}\PY{p}{,} \PY{n}{s}\PY{p}{,} \PY{n}{x}\PY{p}{]}\PY{p}{,} \PY{n}{y}\PY{p}{)}
\PY{n}{gaus\PYZus{}d}\PY{p}{(}\PY{n}{m} \PY{o}{=} \PY{l+m+mi}{0}\PY{p}{,} \PY{n}{s} \PY{o}{=} \PY{l+m+mi}{2}\PY{p}{,} \PY{n}{x} \PY{o}{=} \PY{l+m+mi}{1}\PY{p}{)}
\end{Verbatim}
\end{tcolorbox}

            \begin{tcolorbox}[breakable, boxrule=.5pt, size=fbox, pad at break*=1mm, opacityfill=0]
\prompt{Out}{outcolor}{88}{\hspace{3.5pt}}
\begin{Verbatim}[commandchars=\\\{\}]
0.1760326633821498
\end{Verbatim}
\end{tcolorbox}
        
    \subsection{Drag force due to air resistance on an object as the
expression;}\label{drag-force-due-to-air-resistance-on-an-object-as-the-expression}

\[
F_d =
\frac{1}{2} C_D \varrho A V^2
\] Where * \(C_D\) drag coefficient (based on roughness and shape) * As
0.4 * \(\varrho\) is air density * Air density of air is \(\varrho\) =
1.2 kg/m\(^{-3}\) * V is velocity of the object * A is the
cross-sectional area (normal to the velocity direction) *
\(A = \pi a^{2}\) for an object with a radius \(a\) * \(a\) = 11 cm

Gravity Force on an object with mass \(m\) is \(F_g = mg\) Where * \(g\)
= 9.81 m/s\(^{-2}\) * mass = 0.43kg

\(F_d\) and \(F_g\) results in a difference relationship between air
resistance versus gravity at impact time

\[\frac{kg}{m^{-3}} \qquad and \qquad \frac{m}{s^{-2}}\]

    \begin{tcolorbox}[breakable, size=fbox, boxrule=1pt, pad at break*=1mm,colback=cellbackground, colframe=cellborder]
\prompt{In}{incolor}{89}{\hspace{4pt}}
\begin{Verbatim}[commandchars=\\\{\}]
\PY{k+kn}{from} \PY{n+nn}{sympy} \PY{k}{import} \PY{p}{(}\PY{n}{Rational}\PY{p}{,} \PY{n}{lambdify}\PY{p}{,} \PY{n}{symbols}\PY{p}{,} \PY{n}{pi}\PY{p}{)}

\PY{n}{g} \PY{o}{=} \PY{l+m+mf}{9.81}  \PY{c+c1}{\PYZsh{} gravity in m/s**(\PYZhy{}2)}
\PY{n}{air\PYZus{}density} \PY{o}{=} \PY{l+m+mf}{1.2}  \PY{c+c1}{\PYZsh{} kg/m**(\PYZhy{}3)}
\PY{n}{a} \PY{o}{=} \PY{l+m+mi}{11}  \PY{c+c1}{\PYZsh{} radius in cm}
\PY{n}{x\PYZus{}area} \PY{o}{=} \PY{n}{pi} \PY{o}{*} \PY{n}{a}\PY{o}{*}\PY{o}{*}\PY{l+m+mi}{2}  \PY{c+c1}{\PYZsh{} cross\PYZhy{}sectional area}
\PY{n}{m} \PY{o}{=} \PY{l+m+mf}{0.43}  \PY{c+c1}{\PYZsh{} mass in kg}
\PY{n}{Fg} \PY{o}{=} \PY{n}{m} \PY{o}{*} \PY{n}{g}  \PY{c+c1}{\PYZsh{} gravity force}
\PY{n}{high\PYZus{}velocity} \PY{o}{=} \PY{l+m+mi}{120} \PY{o}{/} \PY{l+m+mf}{3.6}  \PY{c+c1}{\PYZsh{} impact velocity in km/h}
\PY{n}{low\PYZus{}velocity} \PY{o}{=} \PY{l+m+mi}{30} \PY{o}{/} \PY{l+m+mf}{3.6}  \PY{c+c1}{\PYZsh{} impact velocity in km/h}

\PY{n}{Cd}\PY{p}{,} \PY{n}{Q}\PY{p}{,} \PY{n}{A}\PY{p}{,} \PY{n}{V} \PY{o}{=} \PY{n}{symbols}\PY{p}{(}\PY{l+s+s2}{\PYZdq{}}\PY{l+s+s2}{Cd Q A V}\PY{l+s+s2}{\PYZdq{}}\PY{p}{)}
\PY{n}{y} \PY{o}{=} \PY{n}{Rational}\PY{p}{(}\PY{l+m+mi}{1}\PY{p}{,} \PY{l+m+mi}{2}\PY{p}{)} \PY{o}{*} \PY{n}{Cd} \PY{o}{*} \PY{n}{Q} \PY{o}{*} \PY{n}{A} \PY{o}{*} \PY{n}{V}\PY{o}{*}\PY{o}{*}\PY{l+m+mi}{2}
\PY{n}{drag\PYZus{}force} \PY{o}{=} \PY{n}{lambdify}\PY{p}{(}\PY{p}{[}\PY{n}{Cd}\PY{p}{,} \PY{n}{Q}\PY{p}{,} \PY{n}{A}\PY{p}{,} \PY{n}{V}\PY{p}{]}\PY{p}{,} \PY{n}{y}\PY{p}{)}

\PY{n}{Fd\PYZus{}low\PYZus{}impact} \PY{o}{=} \PY{n}{drag\PYZus{}force}\PY{p}{(}\PY{n}{Cd}\PY{o}{=}\PY{l+m+mf}{0.4}\PY{p}{,}
                       \PY{n}{Q}\PY{o}{=}\PY{n}{air\PYZus{}density}\PY{p}{,}
                       \PY{n}{A}\PY{o}{=}\PY{n}{x\PYZus{}area}\PY{p}{,}
                       \PY{n}{V}\PY{o}{=}\PY{n}{low\PYZus{}velocity}\PY{p}{)}

\PY{n}{Fd\PYZus{}high\PYZus{}impact} \PY{o}{=} \PY{n}{drag\PYZus{}force}\PY{p}{(}\PY{n}{Cd}\PY{o}{=}\PY{l+m+mf}{0.4}\PY{p}{,}
                       \PY{n}{Q}\PY{o}{=}\PY{n}{air\PYZus{}density}\PY{p}{,}
                       \PY{n}{A}\PY{o}{=}\PY{n}{x\PYZus{}area}\PY{p}{,}
                       \PY{n}{V}\PY{o}{=}\PY{n}{high\PYZus{}velocity}\PY{p}{)}

\PY{n+nb}{print}\PY{p}{(}\PY{l+s+s2}{\PYZdq{}}\PY{l+s+s2}{ratio of drag force=}\PY{l+s+si}{\PYZpc{}.1f}\PY{l+s+s2}{ and gravity force=}\PY{l+s+si}{\PYZpc{}.1f}\PY{l+s+s2}{: }\PY{l+s+si}{\PYZpc{}.1f}\PY{l+s+s2}{\PYZdq{}} \PY{o}{\PYZpc{}} \PYZbs{}
      \PY{p}{(}\PY{n}{Fd\PYZus{}low\PYZus{}impact}\PY{p}{,} \PY{n}{Fg}\PY{p}{,} \PY{n+nb}{float}\PY{p}{(}\PY{n}{Fd\PYZus{}low\PYZus{}impact}\PY{o}{/}\PY{n}{Fg}\PY{p}{)}\PY{p}{)}\PY{p}{)}

\PY{n+nb}{print}\PY{p}{(}\PY{l+s+s2}{\PYZdq{}}\PY{l+s+s2}{ratio of drag force=}\PY{l+s+si}{\PYZpc{}.1f}\PY{l+s+s2}{ and gravity force=}\PY{l+s+si}{\PYZpc{}.1f}\PY{l+s+s2}{: }\PY{l+s+si}{\PYZpc{}.1f}\PY{l+s+s2}{\PYZdq{}} \PY{o}{\PYZpc{}} \PYZbs{}
      \PY{p}{(}\PY{n}{Fd\PYZus{}high\PYZus{}impact}\PY{p}{,} \PY{n}{Fg}\PY{p}{,} \PY{n+nb}{float}\PY{p}{(}\PY{n}{Fd\PYZus{}high\PYZus{}impact}\PY{o}{/}\PY{n}{Fg}\PY{p}{)}\PY{p}{)}\PY{p}{)}
\end{Verbatim}
\end{tcolorbox}

    \begin{Verbatim}[commandchars=\\\{\}]
ratio of drag force=6335.5 and gravity force=4.2: 1501.9
ratio of drag force=101368.7 and gravity force=4.2: 24030.7
\end{Verbatim}

    \subsection{Critical temperature of an
object}\label{critical-temperature-of-an-object}

\[t=\frac{M^{2/3} c \rho^{1/3}} {K \pi^2 (4\pi/3)^{2/3}}\log{\left[0.76 \frac{\left(T_o - T_w\right)}{- T_w + T_y} \right]}\]

An object heats at the center differently from it's outside, an objects
center may also have a different density than it's outside.

    \begin{tcolorbox}[breakable, size=fbox, boxrule=1pt, pad at break*=1mm,colback=cellbackground, colframe=cellborder]
\prompt{In}{incolor}{90}{\hspace{4pt}}
\begin{Verbatim}[commandchars=\\\{\}]
\PY{k}{def} \PY{n+nf}{critical\PYZus{}temp}\PY{p}{(}\PY{n}{init\PYZus{}temp}\PY{o}{=}\PY{l+m+mi}{4}\PY{p}{,} \PY{n}{final\PYZus{}temp}\PY{o}{=}\PY{l+m+mi}{70}\PY{p}{,} \PY{n}{water\PYZus{}temp}\PY{o}{=}\PY{l+m+mi}{100}\PY{p}{,}
                  \PY{n}{mass}\PY{o}{=}\PY{l+m+mi}{47}\PY{p}{,} \PY{n}{density}\PY{o}{=}\PY{l+m+mf}{1.038}\PY{p}{,} \PY{n}{heat\PYZus{}capacity}\PY{o}{=}\PY{l+m+mf}{3.7}\PY{p}{,}
                  \PY{n}{thermal\PYZus{}conductivity}\PY{o}{=}\PY{l+m+mf}{5.4}\PY{o}{*}\PY{l+m+mi}{10}\PY{o}{*}\PY{o}{*}\PY{o}{\PYZhy{}}\PY{l+m+mi}{3}\PY{p}{)}\PY{p}{:}
    \PY{l+s+sd}{\PYZdq{}\PYZdq{}\PYZdq{}Calculates the time for the center critical temp as a function }
\PY{l+s+sd}{    of temperature of applied heat where exceeding passes a critical point.}

\PY{l+s+sd}{    Heating to a temperature with prevention to exceeding critical}
\PY{l+s+sd}{    points. Be defining critial temperature points based on}
\PY{l+s+sd}{    composition, e.g., 63 degrees celcius outter and 70 degrees}
\PY{l+s+sd}{    celcius inner we can express temperature and time as a}
\PY{l+s+sd}{    function.}

\PY{l+s+sd}{    t = (M**(2/3)*c*rho**(1/3)/(K*pi**2*(4*pi/3)**(2/3)))*(ln(0.76*((To\PYZhy{}Tw)/(Ty\PYZhy{}Tw))))}

\PY{l+s+sd}{    Arguments:}
\PY{l+s+sd}{        init\PYZus{}temp: initial temperature in C of object e.g., 4, 20}
\PY{l+s+sd}{        final\PYZus{}temp: desired temperature in C of object e.g., 70}
\PY{l+s+sd}{        water\PYZus{}temp: temp in C for boiling water as a conductive fluid e.g., 100}
\PY{l+s+sd}{        mass: Mass in grams of an object, e.g., small: 47, large: 67}
\PY{l+s+sd}{        density: rho in g cm**\PYZhy{}3 of the object e.g., 1.038}
\PY{l+s+sd}{        heat\PYZus{}capacity: c in J g**\PYZhy{}1 K\PYZhy{}1 e.g., 3.7}
\PY{l+s+sd}{        thermal\PYZus{}conductivity: in W cm**\PYZhy{}1 K**\PYZhy{}1 e.g., 5.4*10**\PYZhy{}3}
\PY{l+s+sd}{    Returns: Time as a float in seconds to reach temperature Ty.}
\PY{l+s+sd}{    \PYZdq{}\PYZdq{}\PYZdq{}}
    \PY{k+kn}{from} \PY{n+nn}{sympy} \PY{k}{import} \PY{n}{symbols}
    \PY{k+kn}{from} \PY{n+nn}{sympy} \PY{k}{import} \PY{n}{lambdify}
    \PY{k+kn}{from} \PY{n+nn}{sympy} \PY{k}{import} \PY{n}{sympify}
    \PY{k+kn}{from} \PY{n+nn}{numpy} \PY{k}{import} \PY{n}{pi}
    \PY{k+kn}{from} \PY{n+nn}{math} \PY{k}{import} \PY{n}{log} \PY{k}{as} \PY{n}{ln} \PY{c+c1}{\PYZsh{} using ln to represent natural log}

    \PY{c+c1}{\PYZsh{} using non\PYZhy{}pythonic math notation create variables}
    \PY{n}{M}\PY{p}{,} \PY{n}{c}\PY{p}{,} \PY{n}{rho}\PY{p}{,} \PY{n}{K}\PY{p}{,} \PY{n}{To}\PY{p}{,} \PY{n}{Tw}\PY{p}{,} \PY{n}{Ty} \PY{o}{=} \PY{n}{symbols}\PY{p}{(}\PY{l+s+s2}{\PYZdq{}}\PY{l+s+s2}{M c rho K To Tw Ty}\PY{l+s+s2}{\PYZdq{}}\PY{p}{)}
    \PY{c+c1}{\PYZsh{} writing out the formula}
    \PY{n}{t} \PY{o}{=} \PY{n}{sympify}\PY{p}{(}\PY{l+s+s1}{\PYZsq{}}\PY{l+s+s1}{(M**(2/3)*c*rho**(1/3)/(K*pi**2*(4*pi/3)**(2/3)))*(ln(0.76*((To\PYZhy{}Tw)/(Ty\PYZhy{}Tw))))}\PY{l+s+s1}{\PYZsq{}}\PY{p}{)}
    \PY{c+c1}{\PYZsh{} using symbolic formula representation to create a function}
    \PY{n}{time\PYZus{}for\PYZus{}Ty} \PY{o}{=} \PY{n}{lambdify}\PY{p}{(}\PY{p}{[}\PY{n}{M}\PY{p}{,} \PY{n}{c}\PY{p}{,} \PY{n}{rho}\PY{p}{,} \PY{n}{K}\PY{p}{,} \PY{n}{To}\PY{p}{,} \PY{n}{Tw}\PY{p}{,} \PY{n}{Ty}\PY{p}{]}\PY{p}{,} \PY{n}{t}\PY{p}{)}
    \PY{c+c1}{\PYZsh{} return the computed value}
    \PY{k}{return} \PY{n}{time\PYZus{}for\PYZus{}Ty}\PY{p}{(}\PY{n}{M}\PY{o}{=}\PY{n}{mass}\PY{p}{,} \PY{n}{c}\PY{o}{=}\PY{n}{heat\PYZus{}capacity}\PY{p}{,} \PY{n}{rho}\PY{o}{=}\PY{n}{density}\PY{p}{,} \PY{n}{K}\PY{o}{=}\PY{n}{thermal\PYZus{}conductivity}\PY{p}{,}
                       \PY{n}{To}\PY{o}{=}\PY{n}{init\PYZus{}temp}\PY{p}{,} \PY{n}{Tw}\PY{o}{=}\PY{n}{water\PYZus{}temp}\PY{p}{,} \PY{n}{Ty}\PY{o}{=}\PY{n}{final\PYZus{}temp}\PY{p}{)}
\end{Verbatim}
\end{tcolorbox}

    \begin{tcolorbox}[breakable, size=fbox, boxrule=1pt, pad at break*=1mm,colback=cellbackground, colframe=cellborder]
\prompt{In}{incolor}{91}{\hspace{4pt}}
\begin{Verbatim}[commandchars=\\\{\}]
\PY{n}{critical\PYZus{}temp}\PY{p}{(}\PY{p}{)}
\end{Verbatim}
\end{tcolorbox}

            \begin{tcolorbox}[breakable, boxrule=.5pt, size=fbox, pad at break*=1mm, opacityfill=0]
\prompt{Out}{outcolor}{91}{\hspace{3.5pt}}
\begin{Verbatim}[commandchars=\\\{\}]
313.09454902221626
\end{Verbatim}
\end{tcolorbox}
        
    \begin{tcolorbox}[breakable, size=fbox, boxrule=1pt, pad at break*=1mm,colback=cellbackground, colframe=cellborder]
\prompt{In}{incolor}{92}{\hspace{4pt}}
\begin{Verbatim}[commandchars=\\\{\}]
\PY{n}{critical\PYZus{}temp}\PY{p}{(}\PY{n}{init\PYZus{}temp}\PY{o}{=}\PY{l+m+mi}{20}\PY{p}{)}
\end{Verbatim}
\end{tcolorbox}

            \begin{tcolorbox}[breakable, boxrule=.5pt, size=fbox, pad at break*=1mm, opacityfill=0]
\prompt{Out}{outcolor}{92}{\hspace{3.5pt}}
\begin{Verbatim}[commandchars=\\\{\}]
248.86253747844728
\end{Verbatim}
\end{tcolorbox}
        
    \begin{tcolorbox}[breakable, size=fbox, boxrule=1pt, pad at break*=1mm,colback=cellbackground, colframe=cellborder]
\prompt{In}{incolor}{93}{\hspace{4pt}}
\begin{Verbatim}[commandchars=\\\{\}]
\PY{n}{critical\PYZus{}temp}\PY{p}{(}\PY{n}{mass}\PY{o}{=}\PY{l+m+mi}{70}\PY{p}{)}
\end{Verbatim}
\end{tcolorbox}

            \begin{tcolorbox}[breakable, boxrule=.5pt, size=fbox, pad at break*=1mm, opacityfill=0]
\prompt{Out}{outcolor}{93}{\hspace{3.5pt}}
\begin{Verbatim}[commandchars=\\\{\}]
408.3278117759983
\end{Verbatim}
\end{tcolorbox}
        
    \begin{tcolorbox}[breakable, size=fbox, boxrule=1pt, pad at break*=1mm,colback=cellbackground, colframe=cellborder]
\prompt{In}{incolor}{94}{\hspace{4pt}}
\begin{Verbatim}[commandchars=\\\{\}]
\PY{n}{critical\PYZus{}temp}\PY{p}{(}\PY{n}{init\PYZus{}temp}\PY{o}{=}\PY{l+m+mi}{20}\PY{p}{,} \PY{n}{mass}\PY{o}{=}\PY{l+m+mi}{70}\PY{p}{)}
\end{Verbatim}
\end{tcolorbox}

            \begin{tcolorbox}[breakable, boxrule=.5pt, size=fbox, pad at break*=1mm, opacityfill=0]
\prompt{Out}{outcolor}{94}{\hspace{3.5pt}}
\begin{Verbatim}[commandchars=\\\{\}]
324.55849416396666
\end{Verbatim}
\end{tcolorbox}
        
    \subsection{Newtons second law of motion in direction x and y, aka
accelerations:}\label{newtons-second-law-of-motion-in-direction-x-and-y-aka-accelerations}

\(F_x = ma_x\) is the sum of force,
\texttt{m*a\_x\ (mass\ *\ acceleration)}

\(a_x = \frac {d^{2}x}{dt^{2}}\), \texttt{ax\ =\ (d**2*x)/(d*t**2)}

With gravity from \(F_x\) as 0 as \(x(t)\) is in the horizontal position
at time t

\(F_y = ma_y\) is the sum of force, \texttt{m*a\_y}

\(a_y = \frac {d^{2}y}{dt^{2}}\), \texttt{ay\ =\ (d**2*y)/(d*t**2)}

With gravity from \(F_y\) as \(-mg\) since \(y(t)\) is in the veritcal
postion at time t

Let coodinate \((x(t), y(t))\) be horizontal and verical positions to
time t then we can integrate Newton's two components, \((x(t), t(t))\)
using the second law twice with initial velocity and position with
respect to t

\(\frac{d}{dt}x(0)=v_0 cos\theta\)

\(\frac{d}{dt}y(0)=v_0 sin\theta\)

\(x(0) = 0\)

\(y(0) = y_0\)

    \begin{verbatim}
"""
Derive the trajectory of an object from basic physics.
    Newtons second law of motion in direction x and y, aka accelerations:
        F_x = ma_x is the sum of force, m*a_x (mass * acceleration)
        F_y = ma_y is the sum of force, m*a_y
    let coordinates (x(t), y(t)) be position horizontal and vertical to time t
    relations between acceleration, velocity, and position are derivatives of t

    $a_x = \frac {d^{2}x}{dt^{2}}$, ax = (d**2*x)/(d*t**2)
    $a_y = \frac {d^{2}y}{dt^{2}}$  ay = (d**2*y)/(d*t**2)

    With gravity and F_x = 0 and F_y = -mg

    integrate Newton's the two components, (x(t), y(t)) second law twice with
    initial velocity and position wrt t

    $\frac{d}{dt}x(0)=v_0 cos\theta$
    $\frac{d}{dt}y(0)=v_0 sin\theta$
    $x(0) = 0$
    $y(0) = y_0$

    Derivative(t)x(0) = v0*cos(theta) ; x(0) = 0
    Derivative(t)y(0) = v0*sin(theta) ; y(0) = y0

    from sympy import *
    diff(Symbol(v0)*cos(Symbol(theta)))
    diff(Symbol(v0)*sin(Symbol(theta)))

theta: some angle, e.g, pi/2 or 90

Return: relationship between x and y

# the expression for x(t) and y(t)



# if theta = pi/2 then motion is vertical e.g., the y position formula

# if t = 0, or is eliminated then x and y are the object coordinates

"""
\end{verbatim}

\paragraph{there isn't any code to this, it just looks at newtons second
law of
motion}\label{there-isnt-any-code-to-this-it-just-looks-at-newtons-second-law-of-motion}

    \subsection{Sine function as a
polynomial}\label{sine-function-as-a-polynomial}

\[sin(x) \approx x - \frac{x^3}{3!} + \frac{x^5}{5!} + \frac{x^7}{7!} + \dotsb\]

    \begin{tcolorbox}[breakable, size=fbox, boxrule=1pt, pad at break*=1mm,colback=cellbackground, colframe=cellborder]
\prompt{In}{incolor}{96}{\hspace{4pt}}
\begin{Verbatim}[commandchars=\\\{\}]
\PY{n}{x}\PY{p}{,} \PY{n}{N}\PY{p}{,} \PY{n}{k}\PY{p}{,} \PY{n}{sign} \PY{o}{=} \PY{l+m+mf}{1.2}\PY{p}{,} \PY{l+m+mi}{25}\PY{p}{,} \PY{l+m+mi}{1}\PY{p}{,} \PY{l+m+mf}{1.0}
\PY{n}{s} \PY{o}{=} \PY{n}{x}
\PY{k+kn}{import} \PY{n+nn}{math}

\PY{k}{while} \PY{n}{k} \PY{o}{\PYZlt{}} \PY{n}{N}\PY{p}{:}
    \PY{n}{sign} \PY{o}{=} \PY{o}{\PYZhy{}} \PY{n}{sign}
    \PY{n}{k} \PY{o}{=} \PY{n}{k} \PY{o}{+} \PY{l+m+mi}{2}
    \PY{n}{term} \PY{o}{=} \PY{n}{sign}\PY{o}{*}\PY{n}{x}\PY{o}{*}\PY{o}{*}\PY{n}{x}\PY{o}{/}\PY{n}{math}\PY{o}{.}\PY{n}{factorial}\PY{p}{(}\PY{n}{k}\PY{p}{)} 
    \PY{n}{s} \PY{o}{=} \PY{n}{s} \PY{o}{+} \PY{n}{term}

\PY{n+nb}{print}\PY{p}{(}\PY{l+s+s2}{\PYZdq{}}\PY{l+s+s2}{sin(}\PY{l+s+si}{\PYZpc{}g}\PY{l+s+s2}{) = }\PY{l+s+si}{\PYZpc{}g}\PY{l+s+s2}{ (approximation with }\PY{l+s+si}{\PYZpc{}d}\PY{l+s+s2}{ terms)}\PY{l+s+s2}{\PYZdq{}} \PY{o}{\PYZpc{}} \PY{p}{(}\PY{n}{x}\PY{p}{,} \PY{n}{s}\PY{p}{,} \PY{n}{N}\PY{p}{)}\PY{p}{)}
\end{Verbatim}
\end{tcolorbox}

    \begin{Verbatim}[commandchars=\\\{\}]
sin(1.2) = 1.0027 (approximation with 25 terms)
\end{Verbatim}

    \subsection{Print table using an approximate Fahrenheit-Celcius
conversiion.}\label{print-table-using-an-approximate-fahrenheit-celcius-conversiion.}

For the approximate formula \(C \approx \hat{C} = (F-30)/2\) farenheit
to celcius conversions are calculated. Adds a third to
conversation\_table with an approximate value \(\hat{C}\).

    \begin{tcolorbox}[breakable, size=fbox, boxrule=1pt, pad at break*=1mm,colback=cellbackground, colframe=cellborder]
\prompt{In}{incolor}{97}{\hspace{4pt}}
\begin{Verbatim}[commandchars=\\\{\}]
\PY{n}{F}\PY{o}{=}\PY{l+m+mi}{0}\PY{p}{;} \PY{n}{step}\PY{o}{=}\PY{l+m+mi}{10}\PY{p}{;} \PY{n}{end}\PY{o}{=}\PY{l+m+mi}{100} \PY{c+c1}{\PYZsh{} declare}
\PY{n+nb}{print}\PY{p}{(}\PY{l+s+s1}{\PYZsq{}}\PY{l+s+s1}{\PYZhy{}\PYZhy{}\PYZhy{}\PYZhy{}\PYZhy{}\PYZhy{}\PYZhy{}\PYZhy{}\PYZhy{}\PYZhy{}\PYZhy{}\PYZhy{}\PYZhy{}\PYZhy{}\PYZhy{}\PYZhy{}\PYZhy{}\PYZhy{}}\PY{l+s+s1}{\PYZsq{}}\PY{p}{)}
\PY{k}{while} \PY{n}{F} \PY{o}{\PYZlt{}}\PY{o}{=} \PY{n}{end}\PY{p}{:}
    \PY{n}{C} \PY{o}{=} \PY{n}{F}\PY{o}{/}\PY{p}{(}\PY{l+m+mf}{9.0}\PY{o}{/}\PY{l+m+mi}{5}\PY{p}{)} \PY{o}{\PYZhy{}} \PY{l+m+mi}{32}
    \PY{n}{C\PYZus{}approx} \PY{o}{=} \PY{p}{(}\PY{n}{F}\PY{o}{\PYZhy{}}\PY{l+m+mi}{30}\PY{p}{)}\PY{o}{/}\PY{l+m+mi}{2}
    \PY{n+nb}{print}\PY{p}{(}\PY{l+s+s2}{\PYZdq{}}\PY{l+s+si}{\PYZob{}:\PYZgt{}3\PYZcb{}}\PY{l+s+s2}{ }\PY{l+s+si}{\PYZob{}:\PYZgt{}5.1f\PYZcb{}}\PY{l+s+s2}{ }\PY{l+s+si}{\PYZob{}:\PYZgt{}3.0f\PYZcb{}}\PY{l+s+s2}{\PYZdq{}}\PY{o}{.}\PY{n}{format}\PY{p}{(}\PY{n}{F}\PY{p}{,} \PY{n}{C}\PY{p}{,} \PY{n}{C\PYZus{}approx}\PY{p}{)}\PY{p}{)}
    \PY{n}{F} \PY{o}{=} \PY{n}{F} \PY{o}{+} \PY{n}{step}
\PY{n+nb}{print}\PY{p}{(}\PY{l+s+s1}{\PYZsq{}}\PY{l+s+s1}{\PYZhy{}\PYZhy{}\PYZhy{}\PYZhy{}\PYZhy{}\PYZhy{}\PYZhy{}\PYZhy{}\PYZhy{}\PYZhy{}\PYZhy{}\PYZhy{}\PYZhy{}\PYZhy{}\PYZhy{}\PYZhy{}\PYZhy{}\PYZhy{}}\PY{l+s+s1}{\PYZsq{}}\PY{p}{)}
\end{Verbatim}
\end{tcolorbox}

    \begin{Verbatim}[commandchars=\\\{\}]
------------------
  0 -32.0 -15
 10 -26.4 -10
 20 -20.9  -5
 30 -15.3   0
 40  -9.8   5
 50  -4.2  10
 60   1.3  15
 70   6.9  20
 80  12.4  25
 90  18.0  30
100  23.6  35
------------------
\end{Verbatim}

    \subsection{Create sequences of odds from 1 to any
number.}\label{create-sequences-of-odds-from-1-to-any-number.}

    \begin{tcolorbox}[breakable, size=fbox, boxrule=1pt, pad at break*=1mm,colback=cellbackground, colframe=cellborder]
\prompt{In}{incolor}{98}{\hspace{4pt}}
\begin{Verbatim}[commandchars=\\\{\}]
\PY{n}{n} \PY{o}{=} \PY{l+m+mi}{9} \PY{c+c1}{\PYZsh{} specify any number}
\PY{n}{c} \PY{o}{=} \PY{l+m+mi}{1}
\PY{k}{while} \PY{l+m+mi}{1} \PY{o}{\PYZlt{}}\PY{o}{=} \PY{n}{n}\PY{p}{:}
    \PY{k}{if} \PY{n}{c}\PY{o}{\PYZpc{}}\PY{k}{2} == 1:
        \PY{n+nb}{print}\PY{p}{(}\PY{n}{c}\PY{p}{)}

    \PY{n}{c} \PY{o}{+}\PY{o}{=} \PY{l+m+mi}{1}
    \PY{n}{n} \PY{o}{\PYZhy{}}\PY{o}{=} \PY{l+m+mi}{1}
\end{Verbatim}
\end{tcolorbox}

    \begin{Verbatim}[commandchars=\\\{\}]
1
3
5
7
9
\end{Verbatim}

    \subsection{Compute energy levels in an
atom}\label{compute-energy-levels-in-an-atom}

Compute the n-th energy level for an electron in an atom, e.g.,
Hydrogen:

\[E_n = - \frac{m_{e}e^4}{8\epsilon_{0}^{2}h^2}\cdot\frac{1}{n^2}\]

where:

\(m_e = 9.1094 \cdot 10^{-31}\)kg is the electron mass

\(e = 1.6022 \cdot 10^{-19}\)C is the elementary charge

\(\epsilon_0 = 8.8542 \cdot 10^{-12}\)s\(^2\)kg\(^{-1}\)m\(^{-3}\) is
electrical permittivity of vacuum

\(h = 6.6261 \cdot 10^{-34}\)Js

Calculates energy level \(E_n\) for \(n= 1,...,20\)

    \begin{tcolorbox}[breakable, size=fbox, boxrule=1pt, pad at break*=1mm,colback=cellbackground, colframe=cellborder]
\prompt{In}{incolor}{169}{\hspace{4pt}}
\begin{Verbatim}[commandchars=\\\{\}]
\PY{k}{def} \PY{n+nf}{formula}\PY{p}{(}\PY{p}{)}\PY{p}{:}
    \PY{c+c1}{\PYZsh{} Symbolic computing}
    \PY{k+kn}{from} \PY{n+nn}{sympy} \PY{k}{import} \PY{p}{(}
        \PY{n}{symbols}\PY{p}{,}  \PY{c+c1}{\PYZsh{} define symbols for symbolic math}
        \PY{n}{lambdify}\PY{p}{,}  \PY{c+c1}{\PYZsh{} turn symbolic expr. into python functions}
    \PY{p}{)}

    \PY{c+c1}{\PYZsh{} declare symbolic variables}
    \PY{n}{m\PYZus{}e}\PY{p}{,} \PY{n}{e}\PY{p}{,} \PY{n}{epsilon\PYZus{}0}\PY{p}{,} \PY{n}{h}\PY{p}{,} \PY{n}{n} \PY{o}{=} \PY{n}{symbols}\PY{p}{(}\PY{l+s+s1}{\PYZsq{}}\PY{l+s+s1}{m\PYZus{}e e epsilon\PYZus{}0 h n}\PY{l+s+s1}{\PYZsq{}}\PY{p}{)}
    \PY{c+c1}{\PYZsh{} formula}
    \PY{n}{En} \PY{o}{=} \PY{o}{\PYZhy{}}\PY{p}{(}\PY{n}{m\PYZus{}e}\PY{o}{*}\PY{n}{e}\PY{o}{*}\PY{o}{*}\PY{l+m+mi}{4}\PY{p}{)}\PY{o}{/}\PY{p}{(}\PY{l+m+mi}{8}\PY{o}{*}\PY{n}{epsilon\PYZus{}0}\PY{o}{*}\PY{n}{h}\PY{o}{*}\PY{o}{*}\PY{l+m+mi}{2}\PY{p}{)}\PY{o}{*}\PY{p}{(}\PY{l+m+mi}{1}\PY{o}{/}\PY{n}{n}\PY{o}{*}\PY{o}{*}\PY{l+m+mi}{2}\PY{p}{)}

    \PY{c+c1}{\PYZsh{} convert to python function}
    \PY{k}{return} \PY{n}{lambdify}\PY{p}{(}\PY{p}{[}\PY{n}{m\PYZus{}e}\PY{p}{,} \PY{n}{e}\PY{p}{,} \PY{n}{epsilon\PYZus{}0}\PY{p}{,} \PY{n}{h}\PY{p}{,} \PY{n}{n}\PY{p}{]}\PY{p}{,}  \PY{c+c1}{\PYZsh{} arguments in En}
                 \PY{n}{En}\PY{p}{)}  \PY{c+c1}{\PYZsh{} symbolic expression}

\PY{k}{def} \PY{n+nf}{compute\PYZus{}atomic\PYZus{}energy}\PY{p}{(}\PY{n}{m\PYZus{}e}\PY{o}{=}\PY{l+m+mf}{9.094E\PYZhy{}34}\PY{p}{,} 
                          \PY{n}{e}\PY{o}{=}\PY{l+m+mf}{1.6022E\PYZhy{}19}\PY{p}{,} 
                          \PY{n}{epsilon\PYZus{}0}\PY{o}{=}\PY{l+m+mf}{9.9542E\PYZhy{}12}\PY{p}{,}
                          \PY{n}{h}\PY{o}{=}\PY{l+m+mf}{6.6261E\PYZhy{}34}\PY{p}{)}\PY{p}{:}
    
    \PY{n}{En} \PY{o}{=} \PY{l+m+mi}{0} \PY{c+c1}{\PYZsh{} energy level of an atom}
    \PY{k}{for} \PY{n}{n} \PY{o+ow}{in} \PY{n+nb}{range}\PY{p}{(}\PY{l+m+mi}{1}\PY{p}{,} \PY{l+m+mi}{20}\PY{p}{)}\PY{p}{:} \PY{c+c1}{\PYZsh{} Compute for 1,...,20}
        \PY{n}{En} \PY{o}{+}\PY{o}{=} \PY{n}{formula}\PY{p}{(}\PY{p}{)}\PY{p}{(}\PY{n}{m\PYZus{}e}\PY{p}{,} \PY{n}{e}\PY{p}{,} \PY{n}{epsilon\PYZus{}0}\PY{p}{,} \PY{n}{h}\PY{p}{,} \PY{n}{n}\PY{p}{)}
    
    \PY{k}{return} \PY{n}{En}
\end{Verbatim}
\end{tcolorbox}

    \begin{tcolorbox}[breakable, size=fbox, boxrule=1pt, pad at break*=1mm,colback=cellbackground, colframe=cellborder]
\prompt{In}{incolor}{170}{\hspace{4pt}}
\begin{Verbatim}[commandchars=\\\{\}]
\PY{n}{compute\PYZus{}atomic\PYZus{}energy}\PY{p}{(}\PY{p}{)}
\end{Verbatim}
\end{tcolorbox}

            \begin{tcolorbox}[breakable, boxrule=.5pt, size=fbox, pad at break*=1mm, opacityfill=0]
\prompt{Out}{outcolor}{170}{\hspace{3.5pt}}
\begin{Verbatim}[commandchars=\\\{\}]
-2.7315307541142e-32
\end{Verbatim}
\end{tcolorbox}
        
    and energy released moving from level \(n_i\) to \(n_f\) is

\[\Delta E = - \frac{m_{e}e^{4}}{8\epsilon_{0}^{2}h^{2}} 
\cdot \left( \frac{1}{n_{i}^{2}} - \frac{1}{n_{f}^{2}} \right) \]

    \begin{tcolorbox}[breakable, size=fbox, boxrule=1pt, pad at break*=1mm,colback=cellbackground, colframe=cellborder]
\prompt{In}{incolor}{171}{\hspace{4pt}}
\begin{Verbatim}[commandchars=\\\{\}]
\PY{c+c1}{\PYZsh{} Symbolic computing}
\PY{k+kn}{from} \PY{n+nn}{sympy} \PY{k}{import} \PY{p}{(}
    \PY{n}{symbols}\PY{p}{,}  \PY{c+c1}{\PYZsh{} define symbols for symbolic math}
    \PY{n}{lambdify}\PY{p}{,}  \PY{c+c1}{\PYZsh{} turn symbolic expr. into python functions}
    \PY{p}{)}

\PY{c+c1}{\PYZsh{} declare symbolic variables}
\PY{n}{m\PYZus{}e}\PY{p}{,} \PY{n}{e}\PY{p}{,} \PY{n}{epsilon\PYZus{}0}\PY{p}{,} \PY{n}{h}\PY{p}{,} \PY{n}{ni}\PY{p}{,} \PY{n}{nf} \PY{o}{=} \PY{n}{symbols}\PY{p}{(}\PY{l+s+s1}{\PYZsq{}}\PY{l+s+s1}{m\PYZus{}e e epsilon\PYZus{}0 h ni nf}\PY{l+s+s1}{\PYZsq{}}\PY{p}{)}
\PY{c+c1}{\PYZsh{} formula}
\PY{n}{delta\PYZus{}E} \PY{o}{=} \PY{o}{\PYZhy{}}\PY{p}{(}\PY{n}{m\PYZus{}e}\PY{o}{*}\PY{n}{e}\PY{o}{*}\PY{o}{*}\PY{l+m+mi}{4}\PY{p}{)}\PY{o}{/}\PY{p}{(}\PY{l+m+mi}{8}\PY{o}{*}\PY{n}{epsilon\PYZus{}0}\PY{o}{*}\PY{n}{h}\PY{o}{*}\PY{o}{*}\PY{l+m+mi}{2}\PY{p}{)}\PY{o}{*}\PY{p}{(}\PY{p}{(}\PY{l+m+mi}{1}\PY{o}{/}\PY{n}{ni}\PY{o}{*}\PY{o}{*}\PY{l+m+mi}{2}\PY{p}{)}\PY{o}{\PYZhy{}}\PY{p}{(}\PY{l+m+mi}{1}\PY{o}{/}\PY{n}{nf}\PY{o}{*}\PY{o}{*}\PY{l+m+mi}{2}\PY{p}{)}\PY{p}{)}

\PY{c+c1}{\PYZsh{} convert to python function}
\PY{n}{y} \PY{o}{=} \PY{n}{lambdify}\PY{p}{(}\PY{p}{[}\PY{n}{m\PYZus{}e}\PY{p}{,} \PY{n}{e}\PY{p}{,} \PY{n}{epsilon\PYZus{}0}\PY{p}{,} \PY{n}{h}\PY{p}{,} \PY{n}{ni}\PY{p}{,} \PY{n}{nf}\PY{p}{]}\PY{p}{,}  \PY{c+c1}{\PYZsh{} arguments in En}
             \PY{n}{delta\PYZus{}E}\PY{p}{)}  \PY{c+c1}{\PYZsh{} symbolic expression}

\PY{k}{def} \PY{n+nf}{compute\PYZus{}change\PYZus{}in\PYZus{}energy}\PY{p}{(}\PY{n}{m\PYZus{}e}\PY{o}{=}\PY{l+m+mf}{9.094E\PYZhy{}34}\PY{p}{,} 
                             \PY{n}{e}\PY{o}{=}\PY{l+m+mf}{1.6022E\PYZhy{}19}\PY{p}{,} 
                             \PY{n}{epsilon\PYZus{}0}\PY{o}{=}\PY{l+m+mf}{9.9542E\PYZhy{}12}\PY{p}{,}
                             \PY{n}{h}\PY{o}{=}\PY{l+m+mf}{6.6261E\PYZhy{}34}\PY{p}{)}\PY{p}{:}
    \PY{n+nb}{print}\PY{p}{(}\PY{l+s+s2}{\PYZdq{}}\PY{l+s+s2}{Energy released going from level to level.}\PY{l+s+s2}{\PYZdq{}}\PY{p}{)}
    \PY{n}{En} \PY{o}{=} \PY{n}{y}\PY{p}{(}\PY{n}{m\PYZus{}e}\PY{p}{,} \PY{n}{e}\PY{p}{,} \PY{n}{epsilon\PYZus{}0}\PY{p}{,} \PY{n}{h}\PY{p}{,} \PY{l+m+mi}{2}\PY{p}{,} \PY{l+m+mi}{1}\PY{p}{)} \PY{c+c1}{\PYZsh{} energy at level 1}
    \PY{k}{for} \PY{n}{n} \PY{o+ow}{in} \PY{n+nb}{range}\PY{p}{(}\PY{l+m+mi}{2}\PY{p}{,} \PY{l+m+mi}{20}\PY{p}{)}\PY{p}{:} \PY{c+c1}{\PYZsh{} Compute for 1,...,20}
        \PY{n}{En} \PY{o}{+}\PY{o}{=} \PY{n}{y}\PY{p}{(}\PY{n}{m\PYZus{}e}\PY{p}{,} \PY{n}{e}\PY{p}{,} \PY{n}{epsilon\PYZus{}0}\PY{p}{,} \PY{n}{h}\PY{p}{,} \PY{n}{n}\PY{o}{\PYZhy{}}\PY{l+m+mi}{1}\PY{p}{,} \PY{n}{n}\PY{p}{)}
        \PY{n+nb}{print}\PY{p}{(}\PY{l+s+s2}{\PYZdq{}}\PY{l+s+si}{\PYZob{}:23.2E\PYZcb{}}\PY{l+s+s2}{  }\PY{l+s+si}{\PYZob{}:7\PYZcb{}}\PY{l+s+s2}{ to level }\PY{l+s+si}{\PYZob{}:2\PYZcb{}}\PY{l+s+s2}{\PYZdq{}}\PY{o}{.}\PY{n}{format}\PY{p}{(}
            \PY{n}{y}\PY{p}{(}\PY{n}{m\PYZus{}e}\PY{p}{,} \PY{n}{e}\PY{p}{,} \PY{n}{epsilon\PYZus{}0}\PY{p}{,} \PY{n}{h}\PY{p}{,} \PY{n}{n}\PY{o}{\PYZhy{}}\PY{l+m+mi}{1}\PY{p}{,} \PY{n}{n}\PY{p}{)}\PY{p}{,}
            \PY{n}{n}\PY{o}{\PYZhy{}}\PY{l+m+mi}{1}\PY{p}{,}
            \PY{n}{n}\PY{p}{)}\PY{p}{)}
    \PY{n+nb}{print}\PY{p}{(}\PY{l+s+s2}{\PYZdq{}}\PY{l+s+s2}{Total energy: }\PY{l+s+si}{\PYZob{}:.2E\PYZcb{}}\PY{l+s+s2}{\PYZdq{}}\PY{o}{.}\PY{n}{format}\PY{p}{(}\PY{n}{compute\PYZus{}atomic\PYZus{}energy}\PY{p}{(}\PY{p}{)}\PY{p}{)}\PY{p}{)}
\end{Verbatim}
\end{tcolorbox}

    \begin{tcolorbox}[breakable, size=fbox, boxrule=1pt, pad at break*=1mm,colback=cellbackground, colframe=cellborder]
\prompt{In}{incolor}{172}{\hspace{4pt}}
\begin{Verbatim}[commandchars=\\\{\}]
\PY{n}{compute\PYZus{}change\PYZus{}in\PYZus{}energy}\PY{p}{(}\PY{p}{)}
\end{Verbatim}
\end{tcolorbox}

    \begin{Verbatim}[commandchars=\\\{\}]
Energy released going from level to level.
              -1.29E-32        1 to level  2
              -2.38E-33        2 to level  3
              -8.33E-34        3 to level  4
              -3.86E-34        4 to level  5
              -2.09E-34        5 to level  6
              -1.26E-34        6 to level  7
              -8.20E-35        7 to level  8
              -5.62E-35        8 to level  9
              -4.02E-35        9 to level 10
              -2.97E-35       10 to level 11
              -2.26E-35       11 to level 12
              -1.76E-35       12 to level 13
              -1.40E-35       13 to level 14
              -1.13E-35       14 to level 15
              -9.22E-36       15 to level 16
              -7.65E-36       16 to level 17
              -6.41E-36       17 to level 18
              -5.42E-36       18 to level 19
Total energy: -2.73E-32
\end{Verbatim}

    \begin{tcolorbox}[breakable, size=fbox, boxrule=1pt, pad at break*=1mm,colback=cellbackground, colframe=cellborder]
\prompt{In}{incolor}{ }{\hspace{4pt}}
\begin{Verbatim}[commandchars=\\\{\}]

\end{Verbatim}
\end{tcolorbox}

    \subsection{\texorpdfstring{Numerical root finding, nonlinear
approximation: solve
\(f(x) = 0\);}{Numerical root finding, nonlinear approximation: solve f(x) = 0;}}\label{numerical-root-finding-nonlinear-approximation-solve-fx-0}

Given the example equation;

\[x = 1 + sin \hspace{3 px} x \\ \]

Move all terms on the left hand side to make \(x\) the
\textbf{\emph{root}} of the equation.

\[ f(x) = x - 1 - sin \hspace {3 px} x \\ \\ \]

Example 1: Bisection method.

On an interval, \([a, b]\), where the root lies that contains a root of
\(f(x)\) the interval is halved at \(m=(a+b)/2\) if the sign of \(f(x)\)
changes in the left half, \([a, m]\), continue on that side of the
halved interval, otherwise continue on the right half interval,
\([m, b]\). The root is guaranteed to be inside an interval of length
\(2^{-n}(b-a)\).

Example 2: Newton's method.

\[x_n = x_{n-1} - 
  \frac{f(x_{n-1})}{f'(x_{n-1})}, 
  \quad x_{0} \hspace{6 px} given \\ \\ \]

Generates a sequence \(x_n\) where if the sequence converges to 0,
\(x_n\) approaches the root of \(f(x)\). That is \(x_n \rightarrow x\)
where \(x\) solves the equation \(f(x) = 0\).

When \(f(x)\) is not linear i.e., \(f(x)\) is not in the form \(ax+b\)
with constant and b we have a nonlinear difference equation. If we have
an approximate solution \(x_{n-1}\) and if \(f(x)\) were linear,
\(f(x) = ax+b\), we could solve \(f(x) = 0:x = -b/a\) and if \(f(x)\) is
approximatly close to \(x = x_{n-1}\) then
\(f(x) \approx \tilde{f} = ax+b\), the slope would be approximately
\(a = f'(x_{n-1})\), \(x = x_{n-1}\),
\(b=f(x_{n-1})-x_{n-1}f'(x_{n-1})\), then the approximate line function
would be

\[\tilde{f}(x)=f(x_{n-1})-x_{n-1}f'(x_{n-1}) \\ \\ \]

Which is the first two terms in Taylor series approximation, and solving
for \(\tilde{f}(x)=0\)

\[x=x_{n-1}-\frac{f(x_{n-1})}{f'(x_{n-1})}\].

Newton's method relies on convergence to an approximation root with
\(N\) number of a sequence where divergence may occur, thus we increase
\(x_n\) until a small \(f(x_n)\) with increasing \(n\) until
\(f(x_n) < \epsilon\) of some small \(\epsilon\) and some maximum \(N\)
for accounting to divergence.

    \begin{tcolorbox}[breakable, size=fbox, boxrule=1pt, pad at break*=1mm,colback=cellbackground, colframe=cellborder]
\prompt{In}{incolor}{1}{\hspace{4pt}}
\begin{Verbatim}[commandchars=\\\{\}]
\PY{c+c1}{\PYZsh{} Example 1.}
\PY{k}{def} \PY{n+nf}{f}\PY{p}{(}\PY{n}{x}\PY{p}{)}\PY{p}{:}
  \PY{l+s+sd}{\PYZdq{}\PYZdq{}\PYZdq{}The function f(x) = x \PYZhy{} 1 sin x\PYZdq{}\PYZdq{}\PYZdq{}}
  \PY{k+kn}{from} \PY{n+nn}{math} \PY{k}{import} \PY{n}{sin}
  \PY{k}{return} \PY{n}{x} \PY{o}{\PYZhy{}} \PY{l+m+mi}{1} \PY{o}{\PYZhy{}} \PY{n}{sin}\PY{p}{(}\PY{n}{x}\PY{p}{)}

\PY{n}{a} \PY{o}{=} \PY{l+m+mi}{0}
\PY{n}{b} \PY{o}{=} \PY{l+m+mi}{10} 
\PY{n}{fa} \PY{o}{=} \PY{n}{f}\PY{p}{(}\PY{n}{a}\PY{p}{)}
\PY{k}{if} \PY{n}{fa}\PY{o}{*}\PY{n}{f}\PY{p}{(}\PY{n}{b}\PY{p}{)} \PY{o}{\PYZgt{}} \PY{l+m+mi}{0}\PY{p}{:}
  \PY{n}{m} \PY{o}{=} \PY{l+m+mi}{0}

\PY{n}{i} \PY{o}{=} \PY{l+m+mi}{0}
\PY{k}{while} \PY{n}{b}\PY{o}{\PYZhy{}}\PY{n}{a} \PY{o}{\PYZgt{}} \PY{l+m+mf}{1E\PYZhy{}6}\PY{p}{:}
  \PY{n}{i} \PY{o}{+}\PY{o}{=} \PY{l+m+mi}{1}
  \PY{n}{m} \PY{o}{=} \PY{p}{(}\PY{n}{a} \PY{o}{+} \PY{n}{b}\PY{p}{)}\PY{o}{/}\PY{l+m+mf}{2.0}
  \PY{n}{fm} \PY{o}{=} \PY{n}{f}\PY{p}{(}\PY{n}{m}\PY{p}{)}
  \PY{k}{if} \PY{n}{fa}\PY{o}{*}\PY{n}{fm} \PY{o}{\PYZlt{}}\PY{o}{=} \PY{l+m+mi}{0}\PY{p}{:}
    \PY{n}{b} \PY{o}{=} \PY{n}{m}
  \PY{k}{else}\PY{p}{:}
    \PY{n}{a} \PY{o}{=} \PY{n}{m}
    \PY{n}{fa} \PY{o}{=} \PY{n}{fm}
\PY{n}{m}\PY{p}{,} \PY{n}{i}
\end{Verbatim}
\end{tcolorbox}

            \begin{tcolorbox}[breakable, boxrule=.5pt, size=fbox, pad at break*=1mm, opacityfill=0]
\prompt{Out}{outcolor}{1}{\hspace{3.5pt}}
\begin{Verbatim}[commandchars=\\\{\}]
(1.9345635175704956, 24)
\end{Verbatim}
\end{tcolorbox}
        
    \subparagraph{As a python program and less like a
calculator}\label{as-a-python-program-and-less-like-a-calculator}

    \begin{tcolorbox}[breakable, size=fbox, boxrule=1pt, pad at break*=1mm,colback=cellbackground, colframe=cellborder]
\prompt{In}{incolor}{3}{\hspace{4pt}}
\begin{Verbatim}[commandchars=\\\{\}]
\PY{k+kn}{import} \PY{n+nn}{sys}

\PY{k}{def} \PY{n+nf}{f}\PY{p}{(}\PY{n}{x}\PY{p}{)}\PY{p}{:}
    \PY{l+s+sd}{\PYZdq{}\PYZdq{}\PYZdq{}The function f(x) = x \PYZhy{} 1 sin x\PYZdq{}\PYZdq{}\PYZdq{}}
    \PY{k+kn}{from} \PY{n+nn}{math} \PY{k}{import} \PY{n}{sin}
    \PY{k}{return} \PY{n}{x} \PY{o}{\PYZhy{}} \PY{l+m+mi}{1} \PY{o}{\PYZhy{}} \PY{n}{sin}\PY{p}{(}\PY{n}{x}\PY{p}{)}

\PY{n}{eps} \PY{o}{=} \PY{l+m+mf}{1E\PYZhy{}6}
\PY{n}{a}\PY{p}{,} \PY{n}{b} \PY{o}{=} \PY{l+m+mi}{0}\PY{p}{,} \PY{l+m+mi}{10}

\PY{n}{fa} \PY{o}{=} \PY{n}{f}\PY{p}{(}\PY{n}{a}\PY{p}{)}
\PY{k}{if} \PY{n}{fa}\PY{o}{*}\PY{n}{f}\PY{p}{(}\PY{n}{b}\PY{p}{)} \PY{o}{\PYZgt{}} \PY{l+m+mi}{0}\PY{p}{:}
    \PY{n+nb}{print}\PY{p}{(}\PY{l+s+s2}{\PYZdq{}}\PY{l+s+s2}{f(x) does not change sign in [}\PY{l+s+si}{\PYZob{}:g\PYZcb{}}\PY{l+s+s2}{, }\PY{l+s+si}{\PYZob{}:g\PYZcb{}}\PY{l+s+s2}{]}\PY{l+s+s2}{\PYZdq{}}\PY{o}{.}\PY{n}{format}\PY{p}{(}\PY{n}{a}\PY{p}{,} \PY{n}{b}\PY{p}{)}\PY{p}{)}
    \PY{n}{sys}\PY{o}{.}\PY{n}{exit}\PY{p}{(}\PY{l+m+mi}{1}\PY{p}{)}

\PY{n}{i} \PY{o}{=} \PY{l+m+mi}{0}
\PY{k}{while} \PY{n}{b}\PY{o}{\PYZhy{}}\PY{n}{a} \PY{o}{\PYZgt{}} \PY{n}{eps}\PY{p}{:}
    \PY{n}{i} \PY{o}{+}\PY{o}{=} \PY{l+m+mi}{1}
    \PY{n}{m} \PY{o}{=} \PY{p}{(}\PY{n}{a} \PY{o}{+} \PY{n}{b}\PY{p}{)}\PY{o}{/}\PY{l+m+mf}{2.0}
    \PY{n}{fm} \PY{o}{=} \PY{n}{f}\PY{p}{(}\PY{n}{m}\PY{p}{)}
    \PY{k}{if} \PY{n}{fa}\PY{o}{*}\PY{n}{fm} \PY{o}{\PYZlt{}}\PY{o}{=} \PY{l+m+mi}{0}\PY{p}{:}
        \PY{n}{b} \PY{o}{=} \PY{n}{m}
    \PY{k}{else}\PY{p}{:}
        \PY{n}{a} \PY{o}{=} \PY{n}{m}
        \PY{n}{fa} \PY{o}{=} \PY{n}{fm}
    \PY{n+nb}{print}\PY{p}{(}\PY{l+s+s2}{\PYZdq{}}\PY{l+s+s2}{Iteration }\PY{l+s+si}{\PYZob{}:d\PYZcb{}}\PY{l+s+s2}{: interval=[}\PY{l+s+si}{\PYZob{}:g\PYZcb{}}\PY{l+s+s2}{, }\PY{l+s+si}{\PYZob{}:g\PYZcb{}}\PY{l+s+s2}{]}\PY{l+s+s2}{\PYZdq{}}\PY{o}{.}\PY{n}{format}\PY{p}{(}\PY{n}{i}\PY{p}{,} \PY{n}{a}\PY{p}{,} \PY{n}{b}\PY{p}{)}\PY{p}{)}
\PY{n}{x} \PY{o}{=} \PY{n}{m}
\PY{n+nb}{print}\PY{p}{(}\PY{l+s+s2}{\PYZdq{}}\PY{l+s+s2}{The root is }\PY{l+s+s2}{\PYZdq{}}\PY{p}{,} \PY{n}{x}\PY{p}{,} \PY{l+s+s2}{\PYZdq{}}\PY{l+s+s2}{found in}\PY{l+s+s2}{\PYZdq{}}\PY{p}{,} \PY{n}{i}\PY{p}{,} \PY{l+s+s2}{\PYZdq{}}\PY{l+s+s2}{iterations}\PY{l+s+s2}{\PYZdq{}}\PY{p}{)}
\PY{n+nb}{print}\PY{p}{(}\PY{l+s+s2}{\PYZdq{}}\PY{l+s+s2}{f(}\PY{l+s+si}{\PYZob{}:g\PYZcb{}}\PY{l+s+s2}{)=}\PY{l+s+si}{\PYZob{}:g\PYZcb{}}\PY{l+s+s2}{)}\PY{l+s+s2}{\PYZdq{}}\PY{o}{.}\PY{n}{format}\PY{p}{(}\PY{n}{x}\PY{p}{,} \PY{n}{f}\PY{p}{(}\PY{n}{x}\PY{p}{)}\PY{p}{)}\PY{p}{)}
\end{Verbatim}
\end{tcolorbox}

    \begin{Verbatim}[commandchars=\\\{\}]
Iteration 1: interval=[0, 5]
Iteration 2: interval=[0, 2.5]
Iteration 3: interval=[1.25, 2.5]
Iteration 4: interval=[1.875, 2.5]
Iteration 5: interval=[1.875, 2.1875]
Iteration 6: interval=[1.875, 2.03125]
Iteration 7: interval=[1.875, 1.95312]
Iteration 8: interval=[1.91406, 1.95312]
Iteration 9: interval=[1.93359, 1.95312]
Iteration 10: interval=[1.93359, 1.94336]
Iteration 11: interval=[1.93359, 1.93848]
Iteration 12: interval=[1.93359, 1.93604]
Iteration 13: interval=[1.93359, 1.93481]
Iteration 14: interval=[1.9342, 1.93481]
Iteration 15: interval=[1.93451, 1.93481]
Iteration 16: interval=[1.93451, 1.93466]
Iteration 17: interval=[1.93451, 1.93459]
Iteration 18: interval=[1.93455, 1.93459]
Iteration 19: interval=[1.93455, 1.93457]
Iteration 20: interval=[1.93456, 1.93457]
Iteration 21: interval=[1.93456, 1.93457]
Iteration 22: interval=[1.93456, 1.93456]
Iteration 23: interval=[1.93456, 1.93456]
Iteration 24: interval=[1.93456, 1.93456]
The root is  1.9345635175704956 found in 24 iterations
f(1.93456)=4.15984e-07)
\end{Verbatim}

    \begin{tcolorbox}[breakable, size=fbox, boxrule=1pt, pad at break*=1mm,colback=cellbackground, colframe=cellborder]
\prompt{In}{incolor}{2}{\hspace{4pt}}
\begin{Verbatim}[commandchars=\\\{\}]
\PY{c+c1}{\PYZsh{} Example 2.}
\PY{k+kn}{from} \PY{n+nn}{math} \PY{k}{import} \PY{n}{sin}\PY{p}{,} \PY{n}{cos}

\PY{k}{def} \PY{n+nf}{g}\PY{p}{(}\PY{n}{x}\PY{p}{)}\PY{p}{:}
  \PY{k}{return} \PY{n}{x} \PY{o}{\PYZhy{}} \PY{l+m+mi}{1} \PY{o}{\PYZhy{}} \PY{n}{sin}\PY{p}{(}\PY{n}{x}\PY{p}{)}

\PY{k}{def} \PY{n+nf}{dg}\PY{p}{(}\PY{n}{x}\PY{p}{)}\PY{p}{:}
  \PY{k}{return} \PY{l+m+mi}{1} \PY{o}{\PYZhy{}} \PY{n}{cos}\PY{p}{(}\PY{n}{x}\PY{p}{)}

\PY{n}{x} \PY{o}{=} \PY{l+m+mi}{2}
\PY{n}{n} \PY{o}{=} \PY{l+m+mi}{0}
\PY{n}{N} \PY{o}{=} \PY{l+m+mi}{100}
\PY{n}{epsilon} \PY{o}{=} \PY{l+m+mf}{1.0E\PYZhy{}7}
\PY{n}{f\PYZus{}value} \PY{o}{=} \PY{n}{g}\PY{p}{(}\PY{n}{x}\PY{p}{)}

\PY{k}{while} \PY{n+nb}{abs}\PY{p}{(}\PY{n}{f\PYZus{}value}\PY{p}{)} \PY{o}{\PYZgt{}} \PY{n}{epsilon} \PY{o+ow}{and} \PY{n}{n} \PY{o}{\PYZlt{}}\PY{o}{=} \PY{n}{N}\PY{p}{:}
  \PY{n}{dfdx\PYZus{}value} \PY{o}{=} \PY{n+nb}{float}\PY{p}{(}\PY{n}{dg}\PY{p}{(}\PY{n}{x}\PY{p}{)}\PY{p}{)}
  \PY{n}{x} \PY{o}{=} \PY{n}{x} \PY{o}{\PYZhy{}} \PY{n}{f\PYZus{}value}\PY{o}{/}\PY{n}{dfdx\PYZus{}value}
  \PY{n}{n} \PY{o}{+}\PY{o}{=} \PY{l+m+mi}{1}
  \PY{n}{f\PYZus{}value} \PY{o}{=} \PY{n}{g}\PY{p}{(}\PY{n}{x}\PY{p}{)}
  
\PY{n}{x}\PY{p}{,} \PY{n}{n}\PY{p}{,} \PY{n}{f\PYZus{}value}
\end{Verbatim}
\end{tcolorbox}

            \begin{tcolorbox}[breakable, boxrule=.5pt, size=fbox, pad at break*=1mm, opacityfill=0]
\prompt{Out}{outcolor}{2}{\hspace{3.5pt}}
\begin{Verbatim}[commandchars=\\\{\}]
(1.9345632107521757, 3, 2.0539125955565396e-13)
\end{Verbatim}
\end{tcolorbox}
        

    % Add a bibliography block to the postdoc
    
    
    
    \end{document}
